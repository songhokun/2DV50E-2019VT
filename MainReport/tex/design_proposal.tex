This chapter describes the product of the study, considering the analsys made in the previous chapters and given constraints.

\subsection{Non-trusted recursive resolver}
When a trustable recursive resolver does not exist, offen end-users' own computer becomes the recursive resolver. However, the earlier chapter analysed that utilising recursive resolver on a local machine barely gives any value, because many of recursive-to-auth links are unencrypted and subject to the traffic monitoring.

To circumvent the situation, traffic anonymisation technologies on recursive-to auth link can be applied, and examples of such technology are  FreeNet \cite{clarke2001freenet} or GNUNet\cite{grothoff2017gnunet}, and Tor.
However, FreeNet \cite{clarke2001freenet} or GNUNet \cite{grothoff2017gnunet} result in having high delays \cite{anonymousoverdns}.

Solution: Proactive caching \cite{cohen2003proactive} over Tor for most significantly visited websites. For caching, the same tor circut can be reused but for processing individual queries, connection shall not be resued.

Variation of Round Trip Time and its impact on the end-user's perseption shall be discussed. Intercoperation problem with CDN follows.
Provide Privacy analysis on the Tor. Tor may still be subject to the confirmation attacks, similar to DHT technologies. 
