This chapter presents the limitation area of the current DNS Privacy practices.
For the designs using the channel encipherment, the analysis from the previous chapter showed that such designs have insufficient measurements of DNS Queries on Phase 2.
In addition to the limitations of Phase 2, there are several concerns on the measures taken on Phase 1. Following sections introduce privacy concerns on the DNS Privacy enhancing techniques.

\subsection{Privacy leaks by Transitive trust}
Shulman demonstrated that `straightforward application of the encryption alone may not suffice \cite{Shulman:2014}' for protecting DNS Privacy due to possible disruption of DNS Availability and privacy leaks caused by `transitive trust \cite{Ramasubramanian:2005}'.
Shulman further analysed transtive trust as (1) \textit{fan-out} and (2) \textit{chain-length}.
Chain-length refers to ``A number of name servers involved in a resolution of a record that initiates the chain'' and fan-out as ``number of (transitive-trust) chains involved in a resolution of a domain name'' \cite{Shulman:2014}.

Transitive trust at authoritative servers possess a potential risk of revealing the query and client. While DNS QName minimisation \cite{bortzmeyer2016dns} limits the scope of the quername leaks, the extensified usage of DNS in Content Delivery Network (CDN) context \cite{WANG2018235} may have increased the chain-length and it leads to the intercoperation issues \cite{Huque-QNAME-Min-analysis}.
\subsection{Observation on packets' size}

\subsection{Non-trusted recursive resolver}
When a trustable recursive resolver does not exist, offen end-users' own computer becomes the recursive resolver. However, the earlier chapter analysed that utilising recursive resolver on a local machine barely gives any value, because many of recursive-to-auth links are unencrypted and subject to the traffic monitoring.

\subsubsection{Traffic anonymisation}
To circumvent the situation, traffic anonymisation technologies on recursive-to auth link can be applied, and examples of such technology are  FreeNet \cite{clarke2001freenet} or GNUNet\cite{grothoff2017gnunet}, and Tor.
However, FreeNet \cite{clarke2001freenet} or GNUNet \cite{grothoff2017gnunet} result in having high delays \cite{anonymousoverdns}.

Solution: Proactive caching \cite{cohen2003proactive} over Tor for most significantly visited websites. For caching, the same tor circut can be reused but for processing individual queries, connection shall not be resued.

Variation of Round Trip Time and its impact on the end-user's perseption shall be discussed. Intercoperation problem with CDN follows.
Provide Privacy analysis on the Tor. Tor may still be subject to the confirmation attacks, similar to DHT technologies. 

Tor cannot forward UDP traffic on the exit-node. (Citation needed). Therefore, it may be difficult to argure the planned idea.
Instead, try to combine DNS-over-TLS or DNS-over-HTTPS with Tor over its proxy socket \cite{tor-socks}, and try to use a DNS resolving client such as \cite{technitium-configuration}. 