This chapter presents the result of a systematic literature review on studies related to DNS Privacy. Only the studies with Privacy improvement suggestions were chosen to present.
The mitigative methods were classified into three groups, based on the methods' approach to reducing the privacy risks.
%The resonation with the classification choice follows in the next section.

Studies shown in Table \ref{channel} attempted to secure the communication channel of the DNS query.
In other words, these studies suggested applying Encipherment mechanism to deliver Connection Confidentiality as X.800 defines \cite{x800}.
Applying channel encipherment methods for securing DNS queries mitigates the privacy risks by giving trust to the chosen recursive resolver.

\begin{table}[h!]
    \begin{tabular}{ | l | p{10.5cm} | l | l | }
        \hline
            ID & Title & Year & Cites  \\ \hline
            \cite{hu2016specification} & Specification for dns over transport layer security (tls) & 2016 & 27 \\ \hline
            \cite{rfc8484} & Dns queries over https (doh) & 2018 & 5\\ \hline
            \cite{rfc8094} & Dns over datagram transport layer security (dtls) & 2017 & 3\\ \hline
            \cite{bucuti2015opportunistic} & An opportunistic encryption extension for the DNS protocol & 2015 & 2 \\ \hline
            \cite{dickinson2018usage} & Usage profiles for dns over tls and dns over dtls & 2018 & 1 \\ \hline
            \cite{saraj2017design} & Design and implementation of a lightweight privacy extension of DNSSEC protocol & 2017 & 0 \\ \hline
            \cite{dnsoquic} & Specification of DNS over Dedicated QUIC Connections & 2019 & 0 \\ \hline
            \cite{denis2015dnscrypt} & DNSCrypt & 2015 & 0 \\ \hline
            \cite{dempsky2010dnscurve} & DNSCurve & 2009 & 0 \\ \hline
        \end{tabular}
        \caption{Literatures categorised as securing communication channel}
\label{channel}
\end{table}

Table \ref{content} summarised studies on minimising privacy breaching information in the content of packets generated in the DNS query process.
%The approach can be seen as metaphors of Least common mechanism and Isolation as described in security design principles. 

\begin{table}[h!]
    \begin{tabular}{ | l | p{10.5cm} | l | l |}
        \hline
            ID & Title & Year & Cites \\ \hline
            \cite{bortzmeyer2016dns} & DNS query name minimisation to improve privacy & 2016 & 33 \\ \hline
            \cite{annee-dprive-oblivious-dns-00} & Oblivious DNS - Strong Privacy for DNS Queries & 2019 & 0 \\ \hline
            \cite{pan2018mitigating} & Mitigating Client Subnet Leakage in DNS Queries & 2018 & 0 \\ \hline
        \end{tabular}
        \caption{Literatures categorised as securing content}
\label{content}
\end{table}

There are several pieces of research and design proposals of new architecture which would replace the current DNS system. These are found in Table \ref{architectures}

\begin{table}[h!]
    \begin{tabular}{ | l | p{10.5cm} | l | l | }
        \hline
            ID & Title & Year & Cites \\ \hline
            \cite{ambrosin2018security} & Security and privacy analysis of national science foundation future internet architectures & 2018 & 3 \\ \hline
            \cite{grothoff2017gnunet} & The GNUnet System & 2017 & 1 \\ \hline
            \cite{asoni2017paged} & A Paged Domain Name System for Query Privacy & 2017 & 0 \\ \hline
            \cite{loibl2014namecoin} & Namecoin & 2014 & 12 \\ \hline
        \end{tabular}
        \caption{Literatures categorised as Architectural proposal}
    \label{architectures}
\end{table}
\FloatBarrier
\begin{figure}[h!]
    \begin{center}
    \includegraphics*[width=1\columnwidth]{img/dnsprivacy-classification}
    \end{center}
    \caption{Categorisation of DNS Privacy enhancing methods based on the approach}
    \label{dns-methods-classification}
\end{figure}

In addition to the classifications made in the above tables, the identified methods were further sorted into two approaches based on the designers' view of the existing DNS model.
Figure \ref{dns-methods-classification} demonstrated two trees, and of the two trees, the largest tree showed methods which preserve the current hierarchical DNS structure, and the other represented proposals taking a radical approach; suggesting a structural change to decentralised DNS.

Figure \ref{dns-methods-classification} resulted in having two labels of (A) and (C). (A) stood for categorisation proposed in this study and leaves with (C)-label were the methods proposed by other research groups.  

\subsection{Encipherment of communication channels}
Communication channel encrypting methods, as presented in Table \ref{content}, attempted to alleviate privacy risks on the wire.
Methods encrypting the data transport channel were sorted into ones utilising current internet standards or the others proposing customised protocols.

\subsubsection{Reusing the current secure transport protocols}
\textit{DNS-over-TLS (DoT)} \cite{rfc7858}, \textit{DNS-over-HTTPS (DoH)} \cite{rfc8484} and \textit{DNS-over-DTLS} \cite{rfc8094} were the techniques that reuse the proven Internet standards for securing the connection channel between a stub resolver and a recursive resolver.
Technologies in this field could be understood as encapsulations of DNS operations with minimal changes to the existing Internet protocols.

\textit{DoT} and \textit{DoH} applied encryption on the connection-oriented protocol (i.e. TCP), and by the number of citations and client implementations, these were the most prominent technologies within the DNS Privacy field.

In the meantime, \textit{DNS-over-DTLS} attempted to apply encryption over User Datagram Protocol (UDP). In other words, it utilises Data Transport Layer Security (DTLS) \cite{rfc4347}.

\subsubsection{Implementing own transport protocols}
A developing standard of \textit{DNS over Dedicated QUIC} \cite{dnsoquic} also utilised a UDP-like connection called QUIC.
QUIC is ``UDP-Based Multiplexed and Secure Transport \cite{ietf-quic-transport-20}''.
By the time of the study, QUIC was still in a development phase. However, QUIC is prominent to be the next Internet standard as it is on IETF's standardisation track, and the specification is actively being discussed.

\textit{DNSCrypt} \cite{denis2015dnscrypt} and \textit{DNSCurve} \cite{dempsky2010dnscurve} are the non-IETF-standardised techniques that use ``elliptic-curve cryptography (e.g. XSalsa20Poly1305 \cite{chacha} for channel encryption and Curve25519 for key exchange) for enciphering communications. \cite{van2018privacy}''

\subsection{Information redactions}
Applying the encipherment mechanism on stub-recursive communications shifts the trust towards the Recursive resolver.
Design enhancements of packets' content also reduce the risks of Data in the DNS request and data leaks in the servers.
Query Name (QNAME) minimisation \cite{bortzmeyer2016dns} and Oblivious DNS \cite{annee-dprive-oblivious-dns-00} are the techniques of this category.

\textit{QNAME minimisation} reduces the query leaks on higher authoritative name server chains, by presenting only relevant part of a domain name for an authoritative server to answer, instead of querying with FQDN.
It makes only the relevant name server to acquire the full QNAME and query type. Thus, it reduces privacy risks on the recursive-auth link.
A distinction of different links of DNS communication is presented in the next chapter.
The risk of data leak still presents in the last chain of the authoritative NS.

\begin{figure}[h!]
    \begin{center}
    \includegraphics*[width=0.9\columnwidth]{img/ODNSoverview}
    \end{center}
    \caption{An overview of Oblivious DNS \cite{ODNSwebsite}}
    \label{odnsoverview}
\end{figure}
\textit{Oblivious DNS (ODNS)} aimed to decouple any association of a client IP address and DNS query content and no single party should be able to see both \cite{annee-dprive-oblivious-dns-00}.
To decouple client and query name information, using ODNS required two types of servers: normal recursive resolver and ODNS authoritative resolver.
A client with a dedicated stub resolver fetches an encryption key by contacting ODNS server without revealing its desired query, but instead, sending a query of `special.odns'.
The procedure results in creating a unique session key to encrypt its DNS query. The stub resolver appends `odns' TLD to the encrypted query using the key, and the DNS query is sent towards the normal recursive resolver. The recursive resolver queries to one of ODNS authoritative resolvers, which in turn acts as `recursive resolver'.
Figure \ref{odnsoverview} presents an overview of the ODNS, and the illustration is fetched from its project page \cite{ODNSwebsite}.

\subsection{Architectural shift}
%In the previous section, securing recursive-to-auth link is seen challenging due to the hierarchical structure of the DNS.
Studies that aimed towards decentralised DNS system utilised peer-to-peer (P2P) implementation, and the examples of such solution were Namecoin and GNU Name System (GNS).

\textit{Namecoin} \cite{loibl2014namecoin} is described as ``timeline-based system that relies on a P2P network to manage updates and store the timeline \cite{grothoff2017nsa}'', and it settles down any commits related on `key-value mapping' by transactions that are published in an append-only hash chain (a.k.a. blockchain) \cite{kalodner2015empirical}.

\textit{GNS} \cite{grothoff2017nsa, wachs2014censorship} is `privacy-preserving' domain name lookup system which takes a radical deviation from the conventional domain resolvers, by utilising P2P network and Distributed Hash Table (DHT).
DHT provides ``support for an operation: given a key, it maps the key onto a node \cite{stoica2001chord}'', such property is used in GNS domain lookup (resolving) process. 
In other words, GNS utilises ``distributed storage of DNS records in P2P overlay networks \cite{wachs2014censorship}''. In addition to the DHT, GNS is built upon a petname system \cite{stiegler2005introduction} and utilise Simple Distributed Security Infrastructure (SDSI).