This section provides the result of evaluating a possible mitigate option of overcoming limitations by DNS Privacy enhancing methods.
DNS Privacy methods that require architectural shift was excluded in the evaluation.
%Evaluations are made on the solutions that their server and client implementations ready to be used.

Van Heutgen evaluated the combination of using different DNS Privacy techniques in his study \cite{van2018privacy}.
The combinations were a group of channel-enciphering technologies (e.g. DNSCrypt, DoH, and DoT) and QNAME minimisation.
The evaluations of attack surface were introduced based on the location of recursive resolvers: using local recursive resolver, private-remote recursive resolver and the public resolver.

The collective result of his evaluation showed that ``Eavesdroppers who are close to the last chain of the authoritative servers or close to the recursive resolver will be able to obtain a full frame of DNS packet \cite{van2018privacy}''.
Using public resolver may result in third parties eavesdropping or logging DNS communication.
If the cache of the remote-private recursive resolver were not shared among multiple-users, privacy would decrease since the permanent IP address of the remote resolver would be linked to an individual user.

\subsection{Use of Traffic anonymisation}
Apart from van Heutgen’s evaluation, this report further examined the possible combination of using traffic anonymisation techniques and Tor and the combinations as mentioned earlier.
Traffic anonymisation technologies on recursive-to auth link can be applied to circumvent the limitations of channel encryption methods, and examples of such technology are \textit{FreeNet} \cite{clarke2001freenet}, \textit{GNUNet} \cite{grothoff2017gnunet} and \textit{Tor} \cite{dingledine2004tor}.

Tor, the `low-latency' traffic anonymisation tool, encrypts the traffic using multiple envelops and relays the encrypted traffic over three nodes and forwards it to one if its exit nodes.
By this method, the traffic is encrypted and makes the traffic relatively anonymous.
The evaluations are focused on Tor instead of other anonymising services because FreeNet and GNUNet have higher round-trip delay than Tor \cite{anonymousoverdns}.

\subsubsection{Lack of UDP support from Tor}
Using Tor over normal DNS resolution traffic may not be feasible in a conventional way since the majority of DNS traffics are sent over UDP and Tor does not support forwarding UDP traffic to its exit-node \cite{udp-over-tor, dingledine2004tor}.
DNS could also be sent over TCP connection but sending over TCP occurs at limited circumstances where the response exceeds 512 byte-limit \cite{rfc7766}, and the contacted authoritative servers should respect TCP request and reply in TCP packets for the DNS query to work.

\subsubsection{Variation of latency}
Tor employs a two-hop circuit of onion relay servers to increase anonymity of TCP connections between a user and the exit node.
There are several algorithms for selecting relay servers, and Tor by default gives higher weights to relay servers, which are `more longstanding routers and provides higher bandwidth \cite{wacek2013empirical}'.
Since the choice of the relay servers is made using the criteria as mentioned above, it may result in that relay servers are chosen without geographical relevance.
Due to the limitation of the physical medium of communication-link \cite{Singla:2014:ISL}, once relay server in a different continent is selected, it results in having higher latency compared to not using the Tor.
The fact that the choice of relay servers differ on each circuit results in latency variation. In addition to this, ``load-induced queueing contributes significantly to the overall latency variation \cite{Hoiland-Jorgensen:2016}'', and considering that Tor's exit-node process multiple requests, the could be possible bandwidth limitations that result in queueing.
Thus, Applying Tor is considered as latency-expensive.

\subsubsection{Tor on DNS Query Phase 1}
There exist DNS Privacy methods with channel encryption. Despite this, if Tor is chosen to be used in Phase 1, it is regarded as the recursive resolver is not trusted for the risks in the servers.
In this section, evaluations of Tor based on location of recursive resolver follow. Using \textit{local resolver} does not involve Phase 1 over the network, and therefore is not examined.

Using \textit{ISP-provided resolver} over Tor does not add value in protection against attackers near the ISP network, as (1) queries are logged in the servers and (2) DNS transactions on Phase 2 are revealed and (3) although the requester for DNS is diffused the connection can be correlated by session-time observation.

Using \textit{public resolver} over Tor reduces risks of eavesdropping on the wire with Tor's encryption and risks in the servers are also deducted since the recursive resolver only sees the Tor's exit node when ECS is not used.
Eavesdroppers closed to the last chain of authoritative servers persists, they can only induce user as the exit node of a Tor circuit without ECS.
However, it is questioned whether the additional benefit of confidentiality in the recursive resolver server overweights the sacrifice of availability.

\subsubsection{Tor on DNS Query Phase 2}
Proactive caching \cite{cohen2003proactive} over Tor