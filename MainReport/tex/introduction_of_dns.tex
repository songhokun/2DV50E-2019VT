\subsubsection{DNS}\label{dns-introduction}
Every activity on the web most likely begins with entering a human-friendly domain name in the web-browser. Once users enter a domain name for visiting a website, DNS resolves the address to an actual Internet Protocol Address of a web server which hosts the website. In case multiple websites are hosted on a single server, the entered fully qualified domain name(FQDN) is used to differentiate virtual hosts on a web server \cite{virtual24host}. Therefore, DNS is a critical component of the Internet.
%What about describing hjierarchical structure of Domain Name System here?
\subsubsection{DNS Servers}\label{dnsservers}
DNS servers consist of four types: Stub resolver, Recursive resolver, Authoritative server, and Forwarding DNS server. Resolvers refer to programmes that obtain information from name servers upon clients' requests \cite{rfc1034}.

Stub resolver is a resolver that serves as an entry-point of querying DNS from applications and directs search request to the nearest recursive resolver \cite{rfc1123}. As it cannot complete domain name resolution by itself, stub resolver is dependant on a recursive resolver \cite{rfc8499}.

The recursive resolver is a server which receives a DNS query from a stub resolver and gets the final answer to the query, by (1) answering from its local cache or (2) sending queries to other DNS servers \cite{rfc8499}. After a recursive resolver has sent a query request to other authoritative name servers, it is expected for the resolver to store the answer as a \textbf{local cache}. It is the first server in DNS query flow that contacts other servers to get the answer for the client. 

Authoritative (name) server is a server that has ``authority over one or more DNS zones \cite{rfc8499}'' and ``can answer queries without needing to query on other servers as it knows the content of the queried DNS zone by local knowledge \cite{rfc2182}.''

DNS forwarding server is a server that forwards queries to recursive resolver or other forwarding servers. It does not perform a query process for the stub resolver.
\subsubsection{DNS Query process}
Due to the hierarchical structure of the Domain Name System with delegations of authorities \cite{rfc1591}, getting the exact IP address of a given domain name involves several DNS servers. Figure \ref{queryprocess} shows an example of querying ``saimei.ftp.acc.umu.se.''. 
\begin{figure}[ht!]
    \begin{center}
        \includegraphics*[width=\columnwidth]{img/dnsquery}
    \end{center}
    \caption{DNS Query sequence diagram}
    \label{queryprocess}
\end{figure}
\\
In the diagram, steps two and three returns top-level-domain(TLD) from the root servers. The steps four and five obtain the Authoritative name server of Swedish TLD. The .SE TLD returns name server of Ume\aa\ University in steps six and seven. In the last, the name server of umu returns IPv4 address (A record) of the given address, so that recursive resolver can provide the answer to the stub resolver. These steps are performed under the assumption that none of the queries is cached.

\subsubsection{EDNS(0) and Client Subnet}
The extension mechanisms for DNS (EDNS) is specified in RFC 6891.
EDNS allows both DNS servers and the client to send ``larger DNS packet than the original 512 octet limit \cite{rfc6891}'' so that it benefits of utilising larger size.
It makes sending long IPv6 address and possible DNSSEC signatures.
As of February 2019, major DNS resolver operators requires authoritative servers to support EDNS \cite{dns-flag-day, spacek-edns-camel-diet}. 

EDNS(0) provides several options, and one of the options is Client Subnet(ECS) feature, as described in RFC 7831 \cite{rfc7871}. When ECS is used, recursive DNS servers provide a truncated client IP address in its DNS queries to the upstream authorities to permit ``topologically localised answers for Content Delivery Networks (CDN) \cite{kintis2016understanding}''.

\subsubsection{CIA-triad}
In information security discussions, threat mitigations of a system are analysed in three perspectives: confidentiality, availability and integrity. These properties as a group are denoted as CIA triad or the security triad.
Achieving every aspect of CIA-triad often are not feasible, as enhancing one dimension may interfere with the other dimensions \cite{securityincomputing}.
