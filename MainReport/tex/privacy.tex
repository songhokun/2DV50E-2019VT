Privacy is seen as having an access control in confidentiality in a security domain.
In everyday language, Privacy means ``the right to control who knows certain things about you \cite{securityincomputing}''.
When it comes to information security, the special characteristic of information makes a ``propagation problem''.
It means that affected subjects lose control of the information about themselves after being disclosed.

Pfleeger introduces three aspects of Information security: sensitive data, affected parties, and controlled disclosure.
We will discuss privacy issues related to the DNS on different subjects.

\subsection{Affected subjects}
We classify subjects as private persons, and organisations. Organisations are further divided into a large organisation which operates own directory server with DNS, and the smaller organisations that do not operate DNS resolvers in its network.

\subsection{Sensitive information}
Defining what sensitive information is in subjective area.
Therefore, sensitiveness of the information cannot be measured in an absolute scale. However, several common area of sensitive information follows.