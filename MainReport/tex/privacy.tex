Privacy is seen as having an access control in confidentiality in a security domain.
In everyday language, Privacy means ``the right to control who knows certain things about you \cite{securityincomputing}''.
Pfleeger introduces three aspects of Information security: sensitive data, affected parties, and controlled disclosure.
Privacy issues related to the DNS on different subjects will be analysied in this chapter.

\subsection{Affected subjects}
We classify subjects as private persons, and organisations. Organisations are further divided into large organisations that operate own directory server with DNS, and the smaller organisations that do not operate DNS resolvers in its network.

\subsection{Sensitive information}
Defining what sensitive information is in subjective area.
Therefore, sensitiveness of the information cannot be measured in an absolute scale. However, several common area of sensitive information follows.

For natural persons, EU has defined sensitive information as the following: personal data revealing ``(1) ethnic origin, religious or philosophical beliefs, (2) trade-union membership, (3) health-related data, and (4) data concerning a person's sex life \cite{GDPR}''.

For organisations, of their information, assets especially copyright (expression of the idea), trade secret, and privileged information are examples of sensitive information \cite{securityincomputing}.

\subsection{Controlled disclosure}
Information has a different characteristic compared to any physical asset; it can be duplicated to infinite amount at relatively low cost, without harming the asset.
This special characteristic of information makes a ``propagation problem''.
Affected subjects lose control of the information about themselves after being disclosed.

\begin{figure}[ht!]
    \begin{center}
        \includegraphics*[width=\columnwidth]{img/privacyobject}
    \end{center}
    \caption{Privacy related components on DNS transaction}
    \label{privacyobject}
\end{figure}

\subsection{Scenarios}
This section provides analysis of the privacy breaching circumstances per affected subjects by observing DNS. Before focusing impacts on the each subjects, common impacts are instroduced.

Bortzmeyer highighted that ``DNS data itself and a particular transaction'' should be confidential and transaction should not be public \cite{rfc7626}. The transaction refers to the DNS lookup in this context.
Figure \ref{privacyobject} identifies relevant metadata created on DNS transaction. 