From the end user's point of view, recursive DNS Resolvers can be on a local machine, one provided by the Internet Service Providers (ISP) and Public DNS servers \cite{van2018privacy}.
Selection of the location of recursive DNS resolvers leads to different impacts on the user's privacy, in terms of cache-sharing\cite{van2018privacy, wang2013analysis} and obfuscation and logging. Section \ref{dnsservers} described caching on recursive resolvers.

\begin{figure}[ht!]
    \begin{center}
    \includegraphics*[width=0.9\columnwidth]{img/local-recursive}
    \end{center}
    \caption{A simplified network map when using local resursive resolver server.}
    \label{localrecursive}
\end{figure}
When a user utilises a \textit{local recursive resolver} as illustrated in Figure \ref{localrecursive}, channel encipher methods do not add value to users' privacy considering that operations of phase two are often not encrypted.
Supposing that the user does not share local recursive resolver among the others, DNS queries which the user makes will not be fetched from a cache but, instead, from all involved Authoritative Name Servers (NS).
Sending queries in clear text on phase 2 leaves a possibility for all parties who are involved in the network packet transmission to monitor QNAME, query type and source IP of the traffic towards authoritative NS.
Referring to the assumption that local recursive resolver is unique for a person, there is no space for obfuscation since no one else is querying from the IP address of the recursive resolver.
However, utilising the local recursive resolver eliminates the risks of queries being logged (i.e. (3) risk in-the-servers) during Phase 1.


\begin{figure}[h!]
    \begin{center}
    \includegraphics*[width=0.9\columnwidth]{img/isp-recursive}
    \end{center}
    \caption{A simplified network map when using ISP-provided resursive resolver server.}
    \label{isprecursive}
\end{figure}
Using \textit{ISP provided Recursive resolver} is the most common scenario, as most ISP offer DNS resolver to their users by Dynamic Host Configuration Protocol (DHCP).
The resolver from ISP is shared with other subscribers of the network, and it increases more chance of having queries cached by another user who acquired the address previously.
Reusing cache reduces the need of Phase 2 in its response process \cite{wang2013analysis} and thus generates less `often-insecure' traffics towards authoritative NS. 
Authoritative name servers see the source IP address of ISP's resolver in Phase 2 of DNS resolving, but not IP of the individuals, in case E-DNS Client Subset(ECS) is not in place.
When ECS is used, the authoritative NS may see truncated IP of clients \cite{kintis2016understanding}, but the IP address does not present additional privacy harm as ISP's recursive resolver is often in the same subnet IP range, and authoritative NS already acquaints source IP of the recursive DNS resolver. 
Privacy risks incurred by logging may exist in ISP provided resolver, as ISPs may be obliged for log retentions due to legal requirements of the countries they operate in.
Channel encipherment on Phase 1 adds a value of users' privacy to a limited extent, tapping on the wire between the ISP's recursive resolver and a stub resolver is often feasible for ISP itself rather than third parties.

\begin{figure}[h!]
    \begin{center}
    \includegraphics*[width=0.9\columnwidth]{img/public-recursive}
    \end{center}
    \caption{A simplified network map when using public resursive resolver server.}
    \label{publicrecursive}
\end{figure}

A scenario of using a \textit{public recursive resolver} (as known as Public DNS resolver) is represented in Figure \ref{publicrecursive}.
A public DNS server has more possibilities of being shared by a broader public compared to the ISP provided resolvers, and it increases the chances of queries being already cached. Authoritative servers see requests from the IP address of the public resolver, instead of stub resolvers' when ECS is not applied.

This scenario benefits users the most when channel encipherment is applied because DNS query contents in Phase 1 is not visible for parties in the middle of the networking path. It brings significant obfuscation in tracking down the end-user by analysing the network traffics. However, Public DNS servers may log the DNS queries and information of the client. Therefore, privacy risk in-the-servers remains.
