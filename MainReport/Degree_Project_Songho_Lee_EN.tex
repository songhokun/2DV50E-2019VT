%----------------------------------------------------------------------------------------
%
% LaTeX-template for degree projects at LNU, Department of Computer Science
% Last updated by Johan Hagelb\"{a}\ck, Mar 2017
% Linnaeus University
%
% License: Creative Commons BY
%
%----------------------------------------------------------------------------------------

%----------------------------------------------------------------------------------------
%	Settings and configuration
%----------------------------------------------------------------------------------------

\documentclass[a4paper,12pt]{article}

\usepackage[T1]{fontenc}
\usepackage{times}
\usepackage[english,swedish]{babel}
\usepackage[utf8]{inputenc}
\usepackage{dtklogos}
\usepackage{wallpaper}
\usepackage[absolute]{textpos}
\usepackage[top=2cm, bottom=2.5cm, left=3cm, right=3cm]{geometry}
\usepackage{appendix}
\usepackage[nottoc]{tocbibind}
\usepackage[colorlinks=true,
            linkcolor=black,
            urlcolor=blue,
            citecolor=black]{hyperref}

\setcounter{secnumdepth}{3}
\setcounter{tocdepth}{3}

\usepackage{sectsty}
\sectionfont{\fontsize{14}{15}\selectfont}
\subsectionfont{\fontsize{12}{15}\selectfont}
\subsubsectionfont{\fontsize{12}{15}\selectfont}
\usepackage{placeins}
\usepackage{csquotes} % Used to handle citations
\usepackage{pythonhighlight}


\renewcommand{\thetable}{\arabic{section}.\arabic{table}}  
\renewcommand{\thefigure}{\arabic{section}.\arabic{figure}} 

%----------------------------------------------------------------------------------------
%	
%----------------------------------------------------------------------------------------
\newsavebox{\mybox}
\newlength{\mydepth}
\newlength{\myheight}

\newenvironment{sidebar}%
{\begin{lrbox}{\mybox}\begin{minipage}{\textwidth}}%
{\end{minipage}\end{lrbox}%
 \settodepth{\mydepth}{\usebox{\mybox}}%
 \settoheight{\myheight}{\usebox{\mybox}}%
 \addtolength{\myheight}{\mydepth}%
 \noindent\makebox[0pt]{\hspace{-20pt}\rule[-\mydepth]{1pt}{\myheight}}%
 \usebox{\mybox}}

%----------------------------------------------------------------------------------------
%	Title section
%----------------------------------------------------------------------------------------
\newcommand\BackgroundPic{
    \put(-2,-3){
    \includegraphics[keepaspectratio,scale=0.3]{img/lnu_etch.png} % Background picture
    }
}
\newcommand\BackgroundPicLogo{
    \put(30,740){
    \includegraphics[keepaspectratio,scale=0.10]{img/logo.png} % Logo in upper left corner
    }
}

\title{	
\vspace{-8cm}
\begin{sidebar}
    \vspace{10cm}
    \normalfont \normalsize
    \Huge Bachelor Degree Project \\
    \vspace{-1.3cm}
\end{sidebar}
\vspace{3cm}
\begin{flushleft}
    \huge Current milestones of DNS Privacy\\ 
    \it \LARGE - Protection towards pervasive surveillance 
\end{flushleft}
\null
\vfill
\begin{textblock}{6}(10,13)
\begin{flushright}
\begin{minipage}{\textwidth}
\begin{flushleft} \large
\emph{Author:} Songho Lee\\ % Author
\emph{Supervisor:} Ola Flygt\\ % Supervisor
%\emph{Examiner:} Dr.~Mark \textsc{Brown}\\ % Examiner (course manager)
\emph{Semester:} VT 2019\\ % 
\emph{Subject:} Computer Science\\ % Subject area
\end{flushleft}
\end{minipage}
\end{flushright}
\end{textblock}
}

\date{} 
%\linespread{1.25}
\begin{document}
\pagenumbering{gobble}
\newgeometry{left=5cm}
\AddToShipoutPicture*{\BackgroundPic}
\AddToShipoutPicture*{\BackgroundPicLogo}
\maketitle
\restoregeometry
\clearpage
%----------------------------------------------------------------------------------------
%	Abstract
%----------------------------------------------------------------------------------------
\selectlanguage{english}
\begin{abstract}
\noindent Current usage of the DNS system is the most significant loophole of Internet users' privacy, as all queries and answers for resolving web address are not protected in most cases.
The report elaborated which Internet users' privacy interests exist, and presented the current technologies to enhance DNS Privacy through a systematic literature review.
The report also explored the limitations of the current practices and presented several proposals such as DNS-over-Tor and methods to change the trusted recursive resolver to mitigate current limitations periodically.
%Despite a single solution for addressing privacy risks in all stages of the DNS query process does not exist, the report acquaints several Internet standardisations for DNS privacy that are complementary to the existing DNS system and verifies that implementation of these brings significant enhancement of users' privacy.
%The report explores existing methods to enhance DNS Privacy and sets up a series of experiments to verify implementations of such methods for privacy enhancement. % Description of problem explored
\\\\
\textbf{Keywords: DNS, DNS-over-HTTPS, DNS-over-TLS, DNS Privacy}
\end{abstract}

\selectlanguage{swedish}
\begin{abstract}
\noindent Den nuvarande användningen av DNS-systemet är det märkligaste smutthålet för Internetanvändares integritet, eftersom alla förfrågor och svar för att lösa webbadressen inte skyddas i de flesta fall.
Rapporten utarbetade vilka Internetanvändares integritetsintressen som finns och presenterade den nuvarande tekniken för att förbättra DNS-sekretess genom en systematisk litteraturgranskning.
Rapporten undersökte också begränsningarna i den nuvarande praxis och redovisade flera förslag såsom DNS-over-Tor och metoder för att periodiskt ändra den valda pålitliga rekursiva resolvern för att mildra nuvarande begränsningar.
\\\\
\textbf{Nyckelord: DNS, DNS-över-HTTPS, DNS-över-TLS, DNS Sekretess}
\end{abstract}
\selectlanguage{english}
%----------------------------------------------------------------------------------------
%	Preface
%----------------------------------------------------------------------------------------
\newpage
\textbf{\large{Preface}}\\

\noindent I express my sincere appreciation of the good diplomatic relations between the Kingdom of Sweden and the Republic of Korea that had facilitated human resource exchange, with their agreement on working holiday programme (SÖ 2010:22), and generosity of Swedish legislation regarding the tuition fee as dictated in Ordinance (2010: 543) 2 \S \ bullet 4.
These circumstances and help from my beloved parents enabled the financing of the studies.
My gratitude also goes to my manager Patrick and colleagues of UMI team during my time at Ericsson, for having supported me when I decided to finish the degree project.
I cannot forget mentioning my friends and family members who have been emotionally supporting me throughout the process, especially Fangyan, Fengyuan, Jieun and Suryeon. Furthermore, the office door of my supervisor, Ola, has always been open for discussions, and I acknowledge Ola for having helped me whenever I faced difficulties throughout the project.

%----------------------------------------------------------------------------------------
\newpage
\pagenumbering{gobble}
\tableofcontents % Table of contents
\newpage
\pagenumbering{arabic}

%----------------------------------------------------------------------------------------
%
%	Here follows the actual text contents of the report.
%
%----------------------------------------------------------------------------------------

\section{Introduction}
This chapter describes what Doname Name System (DNS) is, and how the legacy design of DNS has become a privacy threat. Before discussing the privacy risks of DNS, the background section introduces relevant structure and mechanisms. Knowledgable readers in DNS and Client subnet function may go to section \ref{problemformulation}.

\subsection{Background}
Digital transformation has brought things used to be done in real life decades ago to the online. At work, people have a video conference call instead of a business trip if not necessary. For shopping goods, people fill in credit card numbers for cross-border payments rather than visiting a bank branch to issue paychecks. In other words, the stage where people work now has been shifted to cyberspace in recent decades, and the Internet has become an essential part of everyday life.

Despite the ubiquitousness of the Internet, users' activities online are collected under pervasive monitoring by different actors.
Pervasive monitoring means ``widespread attack on privacy \cite{rfc7258}.'' Information collected in such action could lead to a breach of users’ privacy, by re-identifying users based on traffic \cite{herrmann2010analyzing}, or could become aids for launching an active form of attacks, such as masquerade and Denial of Service (DoS).
Unfortunately, the insecure architecture of Domain Name System allows the pervasive monitoring, and thus it should be mitigated.
Before discussing the privacy problems of the DNS, DNS and its components are introduced, since these are important to address.

\subsubsection{DNS}
Every activity on the web most likely begins with entering a human-friendly domain name in the web-browser. Once we enter a domain name for visiting a website, DNS resolves the address to an actual Internet Protocol Address of a web server which hosts the website. In case multiple websites are hosted on a single server, the entered fully qualified domain name(FQDN) is used to differentiate virtual hosts on a web server \cite{virtual24host}. Therefore, DNS is a critical component of the Internet.
%What about describing hjierarchical structure of Domain Name System here?
\subsubsection{DNS Servers}\label{dnsservers}
DNS servers consist of four types: Stub resolver, Recursive resolver, Authoritative server, and Forwarding DNS server. Resolvers refer to programmes that obtain information from name servers upon clients' requests \cite{rfc1034}.

Stub resolver is a resolver that serves as an entry-point of querying DNS from applications and directs search request to the nearest recursive resolver \cite{rfc1123}. As it cannot complete domain name resolution by itself, stub resolver is dependant on a recursive resolver \cite{rfc8499}.

The recursive resolver is a server which receives a DNS query from a stub resolver and gets the final answer to the query, by (1) answering from its local cache or (2) sending queries to other DNS servers \cite{rfc8499}. After a recursive resolver has sent a query request to other authoritative name servers, it is expected for the resolver to store the answer as a \textbf{local cache}. It is the first server in DNS query flow that contacts other servers to get the answer for the client. 

Authoritative (name) server is a server that has ``authority over one or more DNS zones \cite{rfc8499}'' and ``can answer queries without needing to query on other servers as it knows the content of the queried DNS zone by local knowledge \cite{rfc2182}.''

DNS forwarding server is a server that forwards queries to recursive resolver or other forwarding servers. It does not perform a query process for the stub resolver.
\subsubsection{DNS Query process}
Due to the hierarchical structure of the Domain Name System with delegations of authorities \cite{rfc1591}, getting the exact IP address of a given domain name involves several DNS servers. Figure \ref{queryprocess} shows an example of querying ``saimei.ftp.acc.umu.se.''. 
\begin{figure}[ht!]
    \begin{center}
        \includegraphics*[width=\columnwidth]{img/dnsquery}
    \end{center}
    \caption{DNS Query sequence diagram}
    \label{queryprocess}
\end{figure}
In the diagram, steps 2 and 3 returns top-level-domain(TLD) from the root servers. The steps 4 and 5 obtain the Authoritative name server of Swedish TLD. The .SE TLD returns name server of Ume\aa\ University in steps 6 and 7. In the last, the name server of umu returns IPv4 address (A record) of the given address, so that recursive resolver can provide the answer to the stub resolver. These steps are performed under the assumption that none of the queries is cached. 
\subsubsection{EDNS(0) and Client Subnet}
The extension mechanisms for DNS (EDNS) is specified in RFC 6891. EDNS allows both DNS servers and client to send ``larger DNS packet than the original 512 octet limit \cite{rfc6891}'' so that it benefits of utilising larger size. It makes sending long IPv6 address and possible DNSSEC signatures. As of February 2019, major DNS resolver operators have started not to support non-EDNS compliant servers. 

EDNS(0) provides several options, and one of the options is Client Subnet(ECS) feature, as described in RFC 7831 \cite{rfc7871}. When ECS is used, recursive DNS servers provide a truncated client IP address in its DNS queries to the upstream authorities to permit ``topologically localised answers for Content Delivery Networks (CDN) \cite{kintis2016understanding}''.

\subsection{Related work}
The project accredits pioneer research of ``DNS Privacy Considerations (RFC 7626) \cite{rfc7626}'' which has provided theoretical foundations in analysis of DNS privacy. 
As of 2019, there are several studies presented to mitigate privacy issues based on the analysis. These studies are later presented in Chapter \ref{surveyresults}.

P. Werneck and J.H.C. van Heugten presented surveys of DNS privacy-enhancing methods. P. Werneck evaluated approaches to improve the privacy of DNS and stated the limitations of identified approaches \cite{werneck2014dns} in 2014. J. Heugten evaluated existing solutions to enhance DNS Privacy in 2018 in his study \cite{van2018privacy}. As standardisation of DNS-over-HTTPS is recently finalised \cite{rfc8484}, a study from van Heugten reflects more recent changes.

\subsection{Problem formulation}\label{problemformulation}
Currently, almost all DNS traffic is sent in clear text \cite{rfc7626} over the UDP protocol \cite{tcp2014analysis}, and it makes DNS queries vulnerable to being hijacked or used to filter users' traffic.
All participants of the DNS query process, as illustrated in Figure \ref{queryprocess}, transmit messages intensively, and these are in plaintext.
It is also noteworthy that all participating authoritative name servers receive the same questions, although it is not necessary for Authoritative servers in higher hierarchies in the process to know all the complete domain address in question.

S. Bortzmeyer has analysed that particular fields in DNS packet \cite{rfc1035} such as Query name (QNAME) and Source IP address reveal ``communication relationships \cite{rfc7626}''.
These series of observations indicate that there are risks of information leakage in following places: (1) tapping on the wire ``between the stub resolvers and the recursive resolvers'', and (2) information leaks in the servers.
The following research questions are formulated, having regards to the privacy breaching circumstances.

\begin{table}[h!]
    \begin{tabular} {|p{1.2cm}|p{12.8cm}|} \hline
        \textbf{RQ1} & Which benefits would DNS Privacy bring to different actors? \\ \hline
        \textbf{RQ2} & Which are the areas that the existing DNS Privacy enhancement methods could not address? \\ \hline
        \textbf{RQ3} & Which combination of technologies would be suitable to address the current limitations?\\ \hline
    \end{tabular}
    \caption{Research questions}
\label{researchquestions}
\end{table}

\subsection{Motivation}
Most of the internet activities begin with DNS query, hence DNS is vital. Notwithstanding the importance of DNS, designers of the current DNS protocol have not taken consideration of ``confidentiality of protocol metadata'' \cite{wachs2014censorship}.
Therefore DNS queries reveal communication flows, and this property of DNS protocol is used in different contexts by different actors. Examples of usages are traffic monitoring for network management or limiting the influence of malicious websites by DNS Footprinting of malware \cite{stoner2010dns}, or detecting malware infections \cite{lemos2013got}.

Other exemplary usages of this property of DNS are nation-state surveillance \cite{NSA-SIGINT}, privacy-unfriendly activities of commercial sectors \cite{weaver2011redirecting}, and illegal actions by criminals. Surveillance affects individuals to possess stress and anxiety \cite{oulasvirta2012long}, and behavioural changes like self-censorship \cite{rfc6973}. RFC 6973 connotes that Privacy harms involve ``harms to financial standing, reputation, solitude, autonomy, and safety \cite{rfc6973}'' of individuals.

S. Farrell et al. state in RFC 7258 that allowing monitoring by benevolent actors and defending privacy against nefarious actors do not hold hand in hand, as the actions required to achieve both, regardless of the motivations, are indistinguishable \cite{rfc7258}.
Disadvantages incurred by lack of DNS privacy significantly overweight advantages, and therefore, DNS privacy should be mitigated in any feasible practices.

\subsection{Objectives}
The following objectives are set to answer the research questions and transform the tasks into the smaller pieces, and reasonings on each objective follow below. Hereafter several words are shortened for the ease of denotation, such as an objective as O and a research question as RQ. Table \ref{objectives} summarises the objectives.
\begin{table}[h!]
    \begin{tabular} {|p{1.2cm}|p{12.8cm}|} \hline
        \textbf{O1} & Investigate different end-users' privacy infringement in scenarios without DNS Privacy\\ \hline
        \textbf{O2} & Explore the status of mitigative methods to enhance DNS Privacy \\ \hline
        \textbf{O3} & Identify areas which the selected methods could not address. \\ \hline
        \textbf{O4} & Explore alternative approaches to mitigating the current limitations.\\ \hline
    \end{tabular}
    \caption{Objectives}
    \label{objectives}
\end{table}

RQ1 aims to explore benefits by having DNS privacy on a different group of people.
As it could be challenging to motivate benefits of having privacy when obscurity prevails, O1 is defined to analyse scenarios when the privacy is violated caused by having insecure or exposed DNS queries.

RQ2 is to find use cases of the existing DNS privacy mitigative methods. However, in order to discuss use cases, the state of these methods need to be examined. Therefore, O2 is set to explore the state of arts of the methods and O3 is set to identify areas for improvements.

RQ3 attempts to provide useful interventions by finding a combination of technologies to overcome the limitation of the current DNS Privacy enhancement. For serving the purpose, O4 discusses possible approaches to using other technologies.

\subsection{Scope/Limitation}
The project has a focus on improving the privacy part, from the security perspectives. In other words, reflected a Confidentiality, Integrity, Availability(CIA) triad, enhancing Integrity and availability perspectives are less prioritised in the current project. Issues and challenges of DNS security as a whole may be found in other studies, such as one conducted by Ning Hu et al.\cite{ning2017dnssecurity}. 

\subsection{Target group}
The project aims to provide insight on DNS Privacy for Internet users and recursive resolver providers for improving users' privacy.
However, it is anticipated that readers have knowledge Internet protocol suite such as TCP/IP \cite{rfc1122}.
%Here you outline which target group that might be interested in your work. If you, for example, do a project about software architectures, a target group can be professional developers and architects that work with similar software systems as the system you investigated.

\subsection{Outline}
The majority of scientific reports follow in Introduction, Methods, Results, Analysis and Discussion(IMRaD) pattern.
This report is also in line with the IMRaD pattern. After the method chapter, four chapters follow, and each chapter presents the result of each objective incrementally. In other words, Chapter 4 presents the result of O1, Chapter 5 presents O2 and it follows up to Chapter 6 which describes the result of O4.

\newpage
\section{Method}\label{Method}

This chapter describes the chosen scientific methods to answer the research questions (Table \ref{researchquestions}) and meet the objectives (Table \ref{objectives}).
The study used scientific methods of systematic literature review and design science.
%Controlled experiments were also partially used to verify several statements made in the study.
Sections follow to motivate the choice of a scientific method for meeting objectives.
\subsection{Systematic literature review}
A systematic literature review was performed to accomplish O2, which is to explore mitigative methods of DNS privacy. Although a recently published article provided an overview of DNS Privacy enhancing methods \cite{van2018privacy}, performing systematic literature review was necessary to eradicate possible biases and to minimise opportunities of missing suitable solutions.

A search criterium was set to list articles that cited RFC 7626 from a database Google Scholar to make the review process systematic. RFC 7626 is chosen, as its analysis had provided a clear insight of DNS privacy issues \cite{rfc7626}, and since around four years had passed after its publication, it was anticipated that fellow researchers have tried to solve or list risks identified in the article.

For inclusion criterium, references of the found articles were further examined, as there were chances of missing to address well-established solutions possibly because the methods had been introduced before publishment of RFC 7626.

Exclusive criteria were set to have the contents of the articles to be relevant to the defined problem. Therefore, any solving other security aspects of DNS, such as availability but not addressing the privacy problems were excluded.
Duplicated entries from the search results were excluded.
In case of having a series of revisions of the same article, the proceeding study was chosen to present.

\subsection{Design Science}
Design science ``creates and evaluates Information Technology artefacts to solve identified organisational problems \cite{von2004design}'', and it has strong relevance in addressing O3 and O4 which are to identify the area that studied DNS Privacy methods could not address and explore alternative approaches to overcome the limitations.

To refine problem statements for the existing practice, a literature review is made to analyse privacy infringement scenarios on diverse user scenarios, as defined in O1.

%\subsection{Controlled Experiment}
%A controlled experiment is applied to verify whether the suggested design artefact addressed the limitations of current DNS Privacy enhancing methods found in O3 or not. Conducting a controlled experiment fulfils one of the guidelines of design science which requires ``thorough evaluation of the artefact \cite{von2004design}''.

\subsection{Reliability and Validity}
The study is seen to have reliability on the results of the literature review, as the same effect will be derived by performing a search as described in the previous section. As Appendix A includes the source code of the experiment scripts, a similar result is expected to be derived by other researchers as well. 

The project deployed its experiments in a virtualised environment to minimise unforeseen factors that would impact performance measurements.

\subsection{Ethical considerations}
Discussing ethical considerations has less significance in the chosen methods, as no real data of any physical persons is collected without the consent.
However, it can be questioned whether it is adequate to elaborate privacy breaching scenarios in details to use the information for formulating problems for the design science method.
Despite the concerns, as the project aims to improve the problematic scenarios of not having sufficient privacy enhancements, describing the problematic situations as-they-are is necessary.

\newpage
\section{Privacy infringement in scenarios}
Privacy is seen as having an access control in confidentiality in a security domain.
In everyday language, Privacy means ``the right to control who knows certain things about you \cite{securityincomputing}''.
When it comes to information security, the special characteristic of information makes a ``propagation problem''.
It means that affected subjects lose control of the information about themselves after being disclosed.

Pfleeger introduces three aspects of Information security: sensitive data, affected parties, and controlled disclosure.
We will discuss privacy issues related to the DNS on different subjects.

\subsection{Affected subjects}
We classify subjects as private persons, and organisations. Organisations are further divided into a large organisation which operates own directory server with DNS, and the smaller organisations that do not operate DNS resolvers in its network.

\subsection{Sensitive information}
Defining what sensitive information is in subjective area.
Therefore, sensitiveness of the information cannot be measured in an absolute scale. However, several common area of sensitive information follows.

For natural persons, legally defined sensitive information are personal data revealing ``(1) ethnic origin, religious or philosophical beliefs, (2) trade-union membership, (3) health-related data, and (4) data concerning a person's sex life \cite{GDPR}''.

For organisations, of their information assets especially copyright (expression of the idea), trade secret, and privileged information may be seen as sensitive information \cite{securityincomputing}.

\subsection{Scenarios}
If a subject and its geographical location Frequencey of visit

\newpage
\section{Status of mitigative methods}\label{surveyresults}
This chapter presents the result of a systematic literature review on studies related to DNS Privacy.
%Afterwards, the analysis of the result follows.
As the raw data from the Google scholar had contained several duplicated entries, duplicated studies were excluded. Series or revisions of the same article are marked as duplicated. For such cases, the proceeding study was chosen to present.
The analysis section which follows after this chapter motivates strategies for categorisation of raw search results.

The refined search results were categorised into the following fields: studies that suggested improving the security breach and investigations that demonstrated any application (or utilisation) of the security vulnerabilities.

\subsection{Improvement suggestions}
Studies shown in Table \ref{channel} attempted to secure the communication channel of the DNS query. In other words, these studies suggested applying Encipherment mechanism to deliver Connection Confidentiality as X.800 defines \cite{x800}.
Applying channel encipherment methods for securing DNS queries mitigates the privacy risks by giving trust to the chosen recursive resolver.

\begin{table}[h!]
    \begin{tabular}{ | l | p{10.5cm} | l | l | }
        \hline
            ID & Title & Year & Cites  \\ \hline
            \cite{hu2016specification} & Specification for dns over transport layer security (tls) & 2016 & 27 \\ \hline
            \cite{rfc8484} & Dns queries over https (doh) & 2018 & 5\\ \hline
            \cite{reddy2017dns} & Dns over datagram transport layer security (dtls) & 2017 & 3\\ \hline
            \cite{bucuti2015opportunistic} & An opportunistic encryption extension for the DNS protocol & 2015 & 2 \\ \hline
            \cite{dickinson2018usage} & Usage profiles for dns over tls and dns over dtls & 2018 & 1 \\ \hline
            \cite{saraj2017design} & Design and implementation of a lightweight privacy extension of DNSSEC protocol & 2017 & 0 \\ \hline
            \cite{dnsoquic} & Specification of DNS over Dedicated QUIC Connections & 2019 & 0 \\ \hline
            \cite{denis2015dnscrypt} & DNSCrypt & 2015 & 0 \\ \hline
            \cite{dempsky2010dnscurve} & DNSCurve & 2009 & 0 \\ \hline
        \end{tabular}
        \caption{Literatures categorised as securing communication channel}
\label{channel}
\end{table}

Table \ref{content} summarised studies on minimising privacy breaching information in the content of packets generated in the DNS query process.
%The approach can be seen as metaphors of Least common mechanism and Isolation as described in security design principles. 

\begin{table}[h!]
    \begin{tabular}{ | l | p{10.5cm} | l | l |}
        \hline
            ID & Title & Year & Cites \\ \hline
            \cite{bortzmeyer2016dns} & DNS query name minimisation to improve privacy & 2016 & 33 \\ \hline
            \cite{annee-dprive-oblivious-dns-00} & Oblivious DNS - Strong Privacy for DNS Queries & 2019 & 0 \\ \hline
            \cite{pan2018mitigating} & Mitigating Client Subnet Leakage in DNS Queries & 2018 & 0 \\ \hline
        \end{tabular}
        \caption{Literatures categorised as securing content}
\label{content}
\end{table}

There are several pieces of research and design proposals of new architecture which would replace the current DNS system. These are found in Table \ref{architectures}

\begin{table}[h!]
    \begin{tabular}{ | l | p{10.5cm} | l | l | }
        \hline
            ID & Title & Year & Cites \\ \hline
            \cite{ambrosin2018security} & Security and privacy analysis of national science foundation future internet architectures & 2018 & 3 \\ \hline
            \cite{grothoff2017gnunet} & The GNUnet System & 2017 & 1 \\ \hline
            \cite{asoni2017paged} & A Paged Domain Name System for Query Privacy & 2017 & 0 \\ \hline
            \cite{loibl2014namecoin} & Namecoin & 2014 & 12 \\ \hline
        \end{tabular}
        \caption{Literatures categorised as Architectural proposal}
    \label{architectures}
\end{table}
\FloatBarrier

\subsection{Categorisation of DNS Privacy enhancing methods}
\begin{figure}[h!]
    \begin{center}
    \includegraphics*[width=1\columnwidth]{img/dnsprivacy-classification}
    \end{center}
    \caption{Categorisation of DNS Privacy enhancing methods based on the approach}
    \label{dns-methods-classification}
\end{figure}

In addition to the classifications made in the above tables, the identified methods were further sorted into two approaches based on the designers' view of the existing DNS model.
Figure \ref{dns-methods-classification} demonstrated two trees, and of the two trees, the largest tree showed methods which preserve the current hierarchical DNS structure, and the other represented proposals taking a radical approach; suggesting a structural change to decentralised DNS.

Figure \ref{dns-methods-classification} resulted in having two labels of (A) and (C). (A) stood for categorisation proposed in this study and leaves with (C)-label were the methods proposed by other research groups.  

\subsection{Encipherment of communication channels}
Communication channel encrypting methods, as presented in in Table \ref{content}, attempted to alleviate privacy risks on the wire.
Methods encrypting the data transport channel were sorted into ones utilising current internet standards or the others proposing customised protocols.

\subsubsection{Reusing the current secure transport protocols}
\textit{DNS-over-TLS(DoT)} \cite{rfc7858}, \textit{DNS-over-HTTPS(DoH)} \cite{rfc8484} and \textit{DNS-over-DTLS} \cite{rfc8094} were the techniques that reuse the proven Internet standards for securing the connection channel between a stub resolver and a recursive resolver.
Technologies in this field could be understood as encapsulations of DNS operations with minimal changes to the existing Internet protocols.

\textit{DoT} and \textit{DoH} applied encryption on the connection-oriented protocol (i.e. TCP), and by the number of citations and client implementations, these were the most prominent technologies within the DNS Privacy field.

In the meantime, \textit{DNS-over-DTLS} attempted to apply encryption over User Datagram Protocol (UDP). In other words, it utilises Data Transport Layer Security (DTLS) \cite{rfc4347}.

\subsubsection{Implementing own transport protocols}
A developing standard of \textit{DNS over Dedicated QUIC} \cite{dnsoquic} also utilised UDP-like connection called QUIC.
QUIC is ``UDP-Based Multiplexed and Secure Transport \cite{ietf-quic-transport-20}''.
By the time of the study, QUIC was still in a development phase. However, QUIC is prominent to be the next Internet standard as it is on IETF's standardisation  track, and the specification is actively being discussed.

\textit{DNSCrypt} \cite{denis2015dnscrypt} and \textit{DNSCurve} \cite{dempsky2010dnscurve} are the non-IETF-standardised techniques that use ``elliptic-curve cryptography (e.g. XSalsa20Poly1305 \cite{chacha} for channel encryption and Curve25519 for key exchange) for enciphering communications. \cite{van2018privacy}''

\subsection{Information redactions}
Applying the encipherment mechanism on stub-recursive communications shifts the trust towards the Recursive resolver.
Design enhancements of packets' content also reduce the risks of Data in the DNS request and data leaks in the servers.
Query Name (QNAME) minimisation \cite{bortzmeyer2016dns} and Oblivious DNS \cite{annee-dprive-oblivious-dns-00} are the techniques of this category.

\textit{QNAME minimisation} reduces the query leaks on higher authoritative name server chains, by presenting only relevant part of a domain name for an authoritative server to answer, instead of querying with FQDN.
It makes only the relevant name server to acquire the full QNAME and query type. Thus, it reduces privacy risks on the recursive-auth link.
A distinction of different links of DNS communication is presented in the next chapter.
The risk of data leak still presents in the last chain of the authoritative NS.

\begin{figure}[h!]
    \begin{center}
    \includegraphics*[width=0.9\columnwidth]{img/ODNSoverview}
    \end{center}
    \caption{An overview of Oblivious DNS \cite{ODNSwebsite}}
    \label{odnsoverview}
\end{figure}
\textit{Oblivious DNS (ODNS)} aimed to decouple any association of a client IP address and DNS query content and no single party should be able to see both \cite{annee-dprive-oblivious-dns-00}.
To decouple client and query name information, using ODNS required two types of servers: normal recursive resolver and ODNS authoritative resolver.
A client with a dedicated stub resolver fetches an encryption key by contacting ODNS server without revealing its desired query, but instead, sending a query of `special.odns'.
The procedure results in creating a unique session key to encrypt its DNS query. The stub resolver appends `odns' TLD to the encrypted query using the key, and the DNS query is sent towards the normal recursive resolver. The recursive resolver queries to one of ODNS authoritative resolvers, which in turn acts as `recursive resolver'.
Figure \ref{odnsoverview} presents an overview of the ODNS and the illustration is fetched from its project page \cite{ODNSwebsite}.

\subsection{Architectural shift}
%In the previous section, securing recursive-to-auth link is seen challenging due to the hierarchical structure of the DNS.
Studies that aimed towards decentralised often imposed peer-to-peer (P2P) implementation, and the examples of such solution were Namecoin and GNU Name System (GNS).

\textit{Namecoin} \cite{loibl2014namecoin} is described as ``timeline-based system that relies on a P2P network to manage updates and store the timeline \cite{grothoff2017nsa}'', and it settles down any commits related on `key-value mapping' by transactions that are published in an append-only hash chain (a.k.a. blockchain) \cite{kalodner2015empirical}.

\textit{GNU Name system (GNS)} \cite{grothoff2017nsa, wachs2014censorship} is `privacy-preserving' domain name lookup system which takes a radical deviation from the conventional domain resolvers, by utilising P2P network and Distributed Hash Table (DHT).
DHT provides ``support for an operation: given a key, it maps the key onto a node \cite{stoica2001chord}'', such property is used in GNS domain lookup (resolving) process. 
In other words, GNS utilise ``distributed storage of DNS records in P2P overlay networks \cite{wachs2014censorship}''. In addition to the DHT, GNS is built upon a petname system \cite{stiegler2005introduction} and utilise Simple Distributed Security Infrastructure (SDSI).


\subsection{Case studies on attack scenarios}
Several studies demonstrated privacy risks of the current DNS standard and proposed mitigative methods which we had introduced in the previous section. 

\begin{table}[h!]
    \begin{tabular}{ | l | p{10.5cm} | l | l | }
        \hline
            ID & Title & Year & Cites \\ \hline
            \cite{kirchler2016tracked} & Tracked without a trace: linking sessions of users by unsupervised learning of patterns in their DNS traffic & 2016 & 10 \\ \hline
            \cite{mohaisen2017leakage} & Leakage of. onion at the DNS Root: Measurements, Causes, and Countermeasures & 2017 & 3 \\ \hline
            \cite{grothoff2017nsa} & NSA's MORECOWBELL: knell for DNS & 2017 & 3 \\ \hline
            \cite{spaulding2018d} & D-FENS: DNS filtering \& extraction network system for malicious domain names & 2018 & 1 \\ \hline
        \end{tabular}
        \caption{Literatures categorised as demonstrating attack scenarios by exploiting the lack of DNS privacy}
\label{attacks}
\end{table}

\newpage
\section{Limitation of DNS Privacy methods}
This section provides limitations of the DNS privacy enhancing methods presented in the previous chapter.
The investigations of the limitations are presented in the same order as the previous categorisation. In the first section, limitations of channel encipherment is introduced.
Afterwards, the focus shifts to information redactions and proposals for structural shifts.

\subsection{Risk areas of DNS Privacy}
DNS Privacy methods could further be analysed based on the which privacy risk these address. S. Bortzmeyer identified risk area of DNS privacy in RFC 7626 as the followings \cite{rfc7626}: 
\begin{enumerate}
    \item Data in the DNS request
    \item On the wire
    \item In the servers
    \item Re-identification and other Interferences
\end{enumerate}

\subsection{Two phases of DNS Query process}
For the conciseness of further analysis, the communication channel (transport channel) of DNS are abstracted into two phases or paths based on the DNS Query process.

\textbf{Phase 1} refers to step 1 and 10 of Figure \ref{queryprocess}. These two steps are performed in the DNS Query process when stub resolver queries a domain name to the recursive resolver and the recursive resolver replies to the stub resolver.
Channel-wise, it is denoted as a \textbf{stub-to-resolver link}.

\textbf{Phase 2} refers to the steps in the Figure \ref{queryprocess} where a recursive resolver finds the final answer to the queried address, by recursively reaching to concerning authoritative servers. In other words, the channel that the rest of the steps is performed is called \textbf{recursive-to-auth link}.

\subsubsection{Insufficient measurements on recursive-to-auth link}\label{recursive-auth-limitations}
The distinction of the two phases are noteworthy, as the current implementations that preserve DNS hierarchical structure do not secure all communication paths towards all involved parties of the DNS resolving.
As an example, the majority of the methods do not encrypt communications on Phase 2 (recursive-to-auth link) in contrast to Phase 1 (stub-to-resolver link).
The fact that an authoritative server having a one-to-many server-client relationship from the recursive resolvers is the major obstacle of applying encryption on Phase 2.

As a further explanation of one-to-many relationship being an obstacle, DoH \cite{rfc8484} and DoT \cite{hu2016specification} use Transport Layer Security (TLS) protocol \cite{rfc7858} for encryption. In the case an authoritative server process multiple TLS session, it is likely to end up exhausting its computational resources \cite{bhople2012server}, similar to a Distributed DoS (DDoS) attack situation. The internet draft ``Next step for DPRIVE: resolver-to-auth link \cite{I-D.bortzmeyer-dprive-step-2}'' discusses the aforementioned challenges.

Furthermore, authentication mechanisms are missing on Phase 2 \cite{I-D.bortzmeyer-dprive-step-2}, and the lack of authentication of the authoritative server may potentially enable a Man-in-the-middle attack (MITM).
% There are proposal of utilising TLS 1.3 and 

\subsubsection{Location of Recursive DNS resolvers}\label{rr-location}
The location of the recursive resolver results in having different attack surfaces, regardless of having applied DNS Privacy methods or not.
In addition to it, the insufficient measurement of recursive-to-auth links makes further complication. Therefore, it is worthy of addressing the location of the resolver to address limitations of the channel encipherment approach.

From the end user's point of view, recursive DNS Resolvers can be on a local machine, one provided by the Internet Service Providers (ISP) and Public DNS servers \cite{van2018privacy}.
Selection of the location of recursive DNS resolvers leads to different impacts on the user's privacy, in terms of cache-sharing\cite{van2018privacy, wang2013analysis} and obfuscation and logging. Section \ref{dnsservers} described caching on recursive resolvers.

\begin{figure}[ht!]
    \begin{center}
    \includegraphics*[width=0.9\columnwidth]{img/local-recursive}
    \end{center}
    \caption{A simplified network map when using local resursive resolver server.}
    \label{localrecursive}
\end{figure}
When a user utilises a \textit{local recursive resolver} as illustrated in Figure \ref{localrecursive}, channel encipher methods do not add value to users' privacy considering that operations of phase two are often not encrypted.
Supposing that the user does not share local recursive resolver among the others, DNS queries which the user makes will not be fetched from a cache but, instead, from all involved Authoritative Name Servers (NS).
Sending queries in clear text on phase 2 leaves a possibility for all parties who are involved in the network packet transmission to monitor QNAME, query type and source IP of the traffic towards authoritative NS.
Referring to the assumption that local recursive resolver is unique for a person, there is no space for obfuscation since no one else is querying from the IP address of the recursive resolver.
However, utilising the local recursive resolver eliminates the risks of queries being logged (i.e. (3) risk in-the-servers) during Phase 1.


\begin{figure}[h!]
    \begin{center}
    \includegraphics*[width=0.9\columnwidth]{img/isp-recursive}
    \end{center}
    \caption{A simplified network map when using ISP-provided resursive resolver server.}
    \label{isprecursive}
\end{figure}
Using \textit{ISP provided Recursive resolver} is the most common scenario, as most ISP offer DNS resolver to their users by Dynamic Host Configuration Protocol (DHCP).
The resolver from ISP is shared with other subscribers of the network, and it increases more chance of having queries cached by another user who acquired the address previously.
Reusing cache reduces the need of Phase 2 in its response process \cite{wang2013analysis} and thus generates less `often-insecure' traffics towards authoritative NS. 
Authoritative name servers see the source IP address of ISP's resolver in Phase 2 of DNS resolving, but not IP of the individuals, in case E-DNS Client Subset(ECS) is not in place.
When ECS is used, the authoritative NS may see truncated IP of clients \cite{kintis2016understanding}, but the IP address does not present additional privacy harm as ISP's recursive resolver is often in the same subnet IP range, and authoritative NS already acquaints source IP of the recursive DNS resolver. 
Privacy risks incurred by logging may exist in ISP provided resolver, as ISPs may be obliged for log retentions due to legal requirements of the countries they operate in \cite{uk-ipa}.
Channel encipherment on Phase 1 adds a value of users' privacy to a limited extent, tapping on the wire between the ISP's recursive resolver and a stub resolver is often feasible for ISP itself rather than third parties.

\begin{figure}[h!]
    \begin{center}
    \includegraphics*[width=0.9\columnwidth]{img/public-recursive}
    \end{center}
    \caption{A simplified network map when using public resursive resolver server.}
    \label{publicrecursive}
\end{figure}

A scenario of using a \textit{public recursive resolver} (as known as Public DNS resolver) is represented in Figure \ref{publicrecursive}.
A public DNS server has more possibilities of being shared by a broader public compared to the ISP provided resolvers, and it increases the chances of queries being already cached. Authoritative servers see requests from the IP address of the public resolver, instead of stub resolvers' when ECS is not applied.

This scenario benefits users the most when channel encipherment is applied because DNS query contents in Phase 1 is not visible for parties in the middle of the networking path. It brings significant obfuscation in tracking down the end-user by analysing the network traffics. However, Public DNS servers may log the DNS queries and information of the client. Therefore, privacy risk in-the-servers remains.

\FloatBarrier

\subsection{Observation on packets' size}
Siby et al. presented their preliminary study results of observing DNS-over-HTTPS traffics \cite{siby2018dns}.
They set two types of experiments: single-query and multi-query. Multi-query referes to simulating to load a web page and let the browser makes follow up queries of the embedded contents of a website.
Their simulation of single-query visit reveals that DNS query-response size tuple results in being distinct per each queried address and when applied multi-query simulation, the number of uniqueness associated to each DNS query-response increased \cite{siby2018dns}.

If more researchers make similar observations, and if the conclusion holds, it becomes a concern since the result indicates that privacy attackers could observe the encrypted query-response set of packets size and induce the actual query despite of not decrypting the packets.

\subsection{Privacy leaks by Transitive trust}
Shulman demonstrated that `straightforward application of the encryption alone may not suffice \cite{Shulman:2014}' for protecting DNS Privacy due to possible disruption of DNS Availability and privacy leaks caused by `transitive trust \cite{Ramasubramanian:2005}'.
Shulman further analysed transtive trust as (1) \textit{fan-out} and (2) \textit{chain-length}.
Chain-length refers to ``A number of name servers involved in a resolution of a record that initiates the chain'' and fan-out as ``number of (transitive-trust) chains involved in a resolution of a domain name'' \cite{Shulman:2014}.

Transitive trust at authoritative servers possess a potential risk of revealing the query and client. While \textit{DNS QName minimisation} \cite{bortzmeyer2016dns} limits the scope of the query name leaks, the extensified usage of DNS in Content Delivery Network (CDN) context \cite{WANG2018235} may have increased the chain-length and it leads to the intercoperation issues \cite{Huque-QNAME-Min-analysis}.

\subsection{Availability concerns on Oblivious DNS}
There are several criteria to define availability. The criteria for defining availability introduced by Pfleeger are (1) having a timely response to a given request, (2) ensuring fair allocation of resources and (3) being fault-tolerant \cite{securityincomputing}.
This report defines availability as a chosen recursive resolver to resolve the queried address and provide the correct IP address promptly.

High availability of DNS is important since web activities cannot be initiated without name resolution (as mentioned in Section \ref{dns-introduction}). High performance (performance refers to having a low latency) of DNS is also an important factor to consider because DNS resolutions are a significant cause of having log Web waits \cite{cohen2003proactive, jung2002dns}, and it has a notable impact on `User-perceived latency'.

However, security concerns arise in terms of availability for \textit{Oblivious DNS}, as ODNS operations require ODNS-authoritative servers in addition to the conventional recursive resolvers and imposing such design creates a new `central point of failure \cite{minutes-102-dprive}'.
Also, questions concerning assuring confidentiality of the keys used in ODNS and issues with fallback have not been answered \cite{minutes-102-dprive}.

\subsection{Prevalence of hierarchical DNS}
Although the primary focus of the study is made on Domain Name resolving part, DNS resolving is only a portion of the entire DNS, as there are diverse actors (such as administrators, domain owners) involved in the system.
Figure \ref{dnsactors}, which is presented by SUNET \cite{SUNET-DNS}, illustrates the involved machines and servers of the system as a whole.

\begin{figure}[h!]
    \begin{center}
    \includegraphics*[width=1\columnwidth]{img/DNS-maskinvara}
    \end{center}
    \caption{Parties involved in DNS as a whole \cite{SUNET-DNS}}
    \label{dnsactors}
\end{figure}

In the figure, sections are vertically divided.
The figure in the first column represents organisations or individuals that desire to acquire a new domain name.
The third column of the figure demonstrates administrators of delegation on each autonomous domain hierarchy.
Legacy of the system are often neglected in attempts of securing the domain name resolution process, which is presented in green colour in the figure. 

The naming hierarchy of the DNS ``ties into systems such as the Public Key Infrastructure (PKI) \cite{akamai-dns-architecture}'', and architecture of decentralised DNS, such as NameCoin \cite{loibl2014namecoin}, may have not considered the PKI structure in its design.

\subsection{Privacy limitations of Namecoin}
Preserving user privacy in Namecoin requires ``replication of the full blockchain at the user's end system \cite{grothoff2017nsa}'' and performing such received a criticism that it ``may be impractical for some devices \cite{grothoff2017nsa}.''
The criticism can be interpreted as applying blockchain did not add significant value, in terms of preserving the privacy of users, as privacy breaches in DNS resolving process could also be addressed if DNS records could be replicated on a client's system even in the existing hierarchical system. 

\subsection{Privacy limitation of GNS}
Authors of the GNS claim that domain resolving process of the GNS is private because ``queries and responses are encrypted \cite{grothoff2017nsa, wachs2014censorship}.''
However, its confidentiality is vulnerable for `confirmation attack' if ``the adversary knows both the public key of the zone and the specific label \cite{wachs2014censorship}.'' Note that the label refers to an entry in a petname system. 

\subsection{Integrity limitations on decentralised approaches}
Ensuring the integrity of the DNS record is significant, and it is evident from `phishing attacks \cite{ariyapperuma2007security, ollmann2004phishing}'.
Per contra, radical protocol design proposals may harm the integrity of the DNS records, as its proof-censorship or legal-attack-proof designs may enable everyone can claim the legitimacy of ownership of the specific domain.

Solutions using DHT in its design may put the system vulnerable to possible Sybil attacks \cite{6503215, SitE2002Scfp}.
Sybil attacks are a type of attack scenario where an attacker `subverts the reputation system by creating a considerable amount of pseudonymous identities \cite{TRIFA20141135}' `in the absence of identification authority \cite{douceur2002sybil}.'

Interventing the malvaceous records are more challenging than the hierarchical structure of the current DNS system in the decentralised design, and this could lead to concerns of enabling phishing attacks.
Enhancing confidentiality aspects of DNS security is important but it should not compromise the integrity aspect.

\subsection{Opportunistic security}
Stenberg, a contributor for client implementation of DNS-over-HTTPS in Firefox, mentioned common Secure-DNS challenges as follows \cite{daniel-doh}: (1) discovery of which recursive resolver to use (2) probing of service (3) opportunistic encryption and (4) blocking force-downgrade.

In the case of Firefox's implementation of DNS-over-HTTPS, the mode of operations are (1) race, (2) Trusted Recursive Resolver (TRR) first, (3) TRR only and (4) shadow mode \cite{firefox-doh-engine}.
Race and shadow mode stands for issuing DNS queries simultaneously on both conventional (insecure) DNS resolving method and Secure DNS method.
In DNS privacy's perspective, these modes are excluded for analysis.

Using TRR first mode attempts to resolve DNS query over DoH and when it fails after pre-set limited of counts, it fallbacks to the conventional DNS resolver.
At the current stage, a preliminary experiment (described in Appendix \ref{protocol-verification}) showed that a limited amount of DNS query still fallbacks to the conventional resolver, and reveals the website a user intended to visit. 

\newpage
\section{Evaluation of alternative approaches}
This section provides the result of evaluating a possible mitigate option of overcoming limitations by DNS Privacy enhancing methods.
DNS Privacy methods that require architectural shift was excluded in the evaluation.
%Evaluations are made on the solutions that their server and client implementations ready to be used.

J. Heugten evaluated the combination of using different DNS Privacy techniques in his study \cite{van2018privacy}.
The combinations were a group of channel-enciphering technologies (e.g. DNSCrypt, DoH, and DoT) and QNAME minimisation.
The evaluations of attack surface were introduced based on the location of recursive resolvers: using local recursive resolver, private-remote recursive resolver and the public resolver.

The collective result of his evaluation showed that ``Eavesdroppers who are close to the last chain of the authoritative servers or close to the recursive resolver will be able to obtain a full frame of DNS packet'', and using public resolver may result in third parties eavesdropping or logging DNS communication \cite{van2018privacy}.
If the cache of the remote-private recursive resolver were not shared among multiple-users, privacy would decrease since the permanent IP address of the remote resolver would be linked to an individual user.

\subsection{Use of Traffic anonymisation}
Apart from J. Heugten’s evaluation, this report further examined the possible combination of using traffic anonymisation techniques and Tor and the combinations as mentioned earlier.
Traffic anonymisation technologies on recursive-to auth link can be applied to circumvent the limitations of channel encryption methods, and examples of such technology are \textit{FreeNet} \cite{clarke2001freenet}, \textit{GNUNet} \cite{grothoff2017gnunet} and \textit{Tor} \cite{dingledine2004tor}.

Tor, the `low-latency' traffic anonymisation tool, encrypts the traffic using multiple envelops and relays the encrypted traffic over three nodes and forwards it to one if its exit nodes.
By this method, the traffic is encrypted and makes the traffic relatively anonymous.
The evaluations are focused on Tor instead of other anonymising services because FreeNet and GNUNet have higher round-trip delay than Tor \cite{anonymousoverdns}.

\subsubsection{Lack of UDP support from Tor}\label{no-udp-tor}
Using Tor over regular DNS resolution traffic may not be feasible in a conventional way since the majority of DNS traffics are sent over UDP and Tor does not support forwarding UDP traffic to its exit-node \cite{udp-over-tor, dingledine2004tor}.
DNS could also be sent over TCP connection but sending over TCP occurs at limited circumstances where the response exceeds 512 byte-limit \cite{rfc7766}, and the contacted authoritative servers should respect TCP request and reply in TCP packets for the DNS query to work.

\subsubsection{Variation of latency}
Tor employs a two-hop circuit of onion relay servers to increase anonymity of TCP connections between a user and the exit node.
There are several algorithms for selecting relay servers, but by default, Tor gives higher weights to relay servers, which are ``more longstanding routers and providing higher bandwidth \cite{wacek2013empirical}.''
Since the choice of the relay servers is made using the criteria mentioned above, it may result in that relay servers are chosen without geographical relevance.
Due to the limitation of the physical medium of communication-link \cite{Singla:2014:ISL}, once relay server of a different continent are selected, it results in having higher latency compared to not using the Tor.
The fact that the choice of relay servers differ on each circuit results in latency variation. In addition to this, ``load-induced queueing contributes significantly to the overall latency variation \cite{Hoiland-Jorgensen:2016}'', and considering that Tor's exit-node process multiple requests, the could be possible bandwidth limitations that result in queueing.
Thus, applying Tor is considered as latency-expensive.

\subsubsection{Tor on DNS Query Phase 1}
There exist DNS Privacy methods with channel encryption. Despite this, if Tor is chosen to be used in Phase 1, it is regarded as the recursive resolver is not trusted for the risks in the servers.
In this section, evaluations of Tor, based on the location of recursive resolver follow. Using \textit{local resolver} does not involve Phase 1 over the network, and therefore is not examined.

Using \textit{ISP-provided resolver} over Tor does not add value in protection against attackers near the ISP network, as (1) queries are logged in the servers and (2) DNS transactions on Phase 2 are revealed and (3) although the requester for DNS is diffused the connection can be correlated by session-time observation.

Using \textit{public resolver} over Tor reduces risks of eavesdropping on the wire with Tor's encryption and risks in the servers are also deducted since the recursive resolver only sees the Tor's exit node when ECS is not used.
Eavesdroppers closed to the last chain of authoritative servers persists, but they can only induce user as the exit node of a Tor circuit without ECS.
However, it is questioned whether the additional benefit of confidentiality in the recursive resolver server overweights the sacrifice of availability.

\subsubsection{Tor on DNS Query Phase 2}
As previously mentioned, applying Tor on DNS queries is costly.
While DNS channel encryption methods had limitations on Phase 2, applying anonymity which Tor provides on Phase 2 of DNS query could be beneficial.
However, having control of Phase 2 limits the choice of the recursive resolver.
For individuals, it indicates using the local recursive resolver.
For a group of individuals or organisations, it would indicate that outgoing DNS queries from the managed recursive resolver towards authoritative Name servers are anonymised using the Tor network.

Despite the variation of latency imposed by Tor circuit, the performance degradation caused by Tor in Phase 2 could be mitigated by the means of `Proactive caching \cite{cohen2003proactive}', which is actively managing the cache stored on the recursive resolver rather than relying on passively created caches that die after their Time-to-Live (TTL) period.

\subsection{Use of multiple trusted recursive resolvers}
This section describes a proposal to address the privacy risk of reidentification and other interferences.
Regarding the reidentification risk, section \ref{fingerprint} introduced a potential privacy violation caused by `behaviour-based tracking' which is proposed by Kirchler et al. \cite{kirchler2016tracked}.

Kirchler's study had a focus on introducing tracking methods using an unsupervised learning technique and suggested that users could circumvent such tracking attempts by changing their IP address frequently.
Although their study did not mention in specific how the researchers obtained the logs related to the DNS queries of various clients, it is anticipated that once DNS privacy-enhancing methods are in place, the most probable site of privacy violation would be inside the DNS recursive resolver.

To overcome the risks, this section delegate tasks into two group of administrators: DNS Resolver admin and local network admin.
If the DNS resolver is not managed within the same organisation, the recommendations of section \ref{dnsresolveradmin} can be ignored.

\subsubsection{Actions required on local DNS Resolver admin}\label{dnsresolveradmin}
Administrators could prepare a list of the trusted recursive resolver (TRR) based on log retention policy and deployment of QName minimisation.
Once the list is prepared, they could configure the local server to forward DNS queries outside the organisation to ones from the list of TRRs.
The selection of designated forwarder could be changed in a certain time interval, such as every six hours.

If it is desirable to obfuscate the frequency of visit or temporal information of visiting websites, DNS resolver admin could choose to manage DNS cache actively, by means of the `proactive caching \cite{cohen2003proactive}'. Alternatively, performing `multi-queries \cite{siby2018dns}' of the frequently visited websites, such as following Alexa Top lists, could be done to maximise cache-sharing effect \cite{wang2013analysis}. Appendix \ref{scriptcode-appendix} could be referred for achieving this.

\subsubsection{Actions required on Network admin}
DPRIVE workgroup of IETF has on-going standardisation tasks of ``DoT announcements using DHCP or Router Advertisements \cite{peterson-dot-dhcp-00}'' and ``DoH resolver announcement Using DHCP or Router Advertisements \cite{peterson-doh-dhcp-00}''.
Once the standardisation process is finalised and the networking industries implement the new standards in their products, local network administrators can configure in the DHCP server to direct which Secure DNS resolver to use.

Once network administrators change DHCP configuration to redirect clients to use different sets of TRR at a certain time interval of choice, it is anticipated that end-clients' TRR configurations would be changed according to the pre-set DHCP lease time. The network administrators could consider the lease time for periodically changing the trusted DNS recursive resolver. However, there is no study at the moment that has verified the current proposal.
\newpage

\section{Discussion}
The project intensively examined securing Domain Name queries as a method of enhancing end-users' privacy towards pervasive monitoring and presented sections for answering the defined objectives. In this chapter, arguments on the current direction of DNS Privacy methods developments are introduced. Arguments of whether it is ethical to empower DNS Privacy also follows.

\subsection{Privacy leaking components apart from the DNS}
A question may arise why it mainly focuses on securing DNS, although there exist other factors which disclose users' privacy.
Mechanisms of a web filter are examined to answer the question, as it is a commonly found practical example of the large scale monitoring \cite{murdoch2008tools}.

Web filter, also known as content-control software, is software that restricts access to a content that is delivered on the Web.
Wazen et. al categorise mechanisms of legacy web-filtering into five techniques: (a) Port-based, (b) DNS, (c) IP Address, (d) Certificate, (e) Payload-based (f) HTTP proxy filtering techniques \cite{shbair2015efficiently}.
Except for the technique based on DNS filtering, the rest methods are regarded as well-mitigated due to recent developments of the web environment. 

Among the various types of filtering techniques mentioned above, methods (a) and (c) are considered less efficient due to changes in the Internet ecosystem in recent decennial;
Internet firms such as Google, Facebook and Amazon show strong presence \cite{haucap2014google}, and the phenomenon may have reduced the diversity of traffic endpoint's IP addresses.
Moreover, it has become more common to have web services deployed in cloud environments \cite{clouds2018stat}, and IaaS providers extensively use `Virtual Host \cite{virtual24host}', which means various Web servers correspond to the same IP address.
It also eliminates the need for utilising different ports to co-host services. Thus, port usages are normalised.

Also, another notable change of the Internet is that adoption of HTTPS on the web has increased significantly \cite{felt2017measuring}.
The change has increased costs of performing technique (e) and brought challenges in payload-based traffic classification \cite{xue2013traffic}.
Also, it has made (f) less applicable, as a proxy does not directly process encrypted traffics \cite{shbair2015efficiently}.
Furthermore, the combination of wide deployment of HTTPS and Virtual Hosting has made technique (e) inefficient, because ``many companies share the same certificate across different services and domain names \cite{shbair2015efficiently}''.

However, the trend change of Internet has not brought additional challenges to Domain Name System (DNS) filtering. Therefore, the project studies to remedy the weakest point towards users' privacy, which in this case, is DNS.

\subsection{DNS Privacy - possible aids for the criminals}
At the beginning of the report, the thesis motivated that securing DNS in its architecture is necessary, as the vulnerability cannot only be exploited by the right hand but could also be misused by a malevolent party.
It is feasible that the perpetrators may use DNS Privacy enhancement methods to hide their activities, and DNS Privacy is in good practice that makes it difficult for investigators to eavesdrop the queries.

In the U.S. or common law jurisdictions, hindering lawful enforcement may be seen as ``Obstruction of justice''.
In this perspective, hiding DNS query by means of encryptions could be questioned whether such behaviour is \textit{concealing} or \textit{covering up} with the intent to \textit{impede} or \textit{obstruct} the investigation or proper administration as 18 U.S. Code \S 1591 states \cite{Obstructionofjustice}.

M Bay has performed a thought experiment of the relation of civil disobedience and unbreakable encryption \cite{bay2017ethics} based on John Rawls' theory of justice.
Not all DNS Privacy enhancing methods have direct linkage with the `unbreakable encryption' since encryptions are breakable with `the given sufficient time and computational resources \cite{ellison2000ten,chau2006application}'.
However, it is noteworthy to introduce the aspect as exercising DNS Privacy has similarity in a sense that DNS Privacy methods aim to achieve unbreakable-encryption-like situation.

M Bay had made three assumptions before applying Rawlsian principles to encryption as follows \cite{bay2017ethics}:
\begin{enumerate}
  \item Unbreakable/impenetrable encryption is indeed impenetrable
  \item The encryption in question is available to citizens and is not exclusive to certain institutions within society
  \item The society examined here can be described as well-ordered \cite{moon_2014, RawlsJohn1973Atoj} in Rawls’ terminology
\end{enumerate}
The thought experiment led to the following points: ``In a well-ordered society, an obstruction of law enforcement is at the same time an obstruction of principles of justice agreed upon by the society’s citizens.'' and  ``if society is just, citizens must comply with its institutions in order for it to remain just.''
However, in the case of ``societies are only partially just or in which, say, an unjust war is waged by an otherwise just society. Then, the basic liberties of the individual take precedence, in the push for the restoration of justice \cite{RawlsJohn1973Atoj}.''
However, according to Bay, civil disobedience and conscientious refusal lead to a violation of Rawls’ principles as he hypothesised with point three: the society is well-ordered.

With regards to applying utilitarian arguments for encryption, M Bay concluded as ``utilitarian reasoning does not provide us with a solution with regard to the conflict between encryption, privacy and enforcement of justice, since it simply becomes a version of the age-old conflict between security and freedom at a higher level of abstraction \cite{bay2017ethics}.''

Judging whether empowering DNS privacy is justice or not is a debatable question in many aspects. However, the encryption or having privacy facilities the freedom of speech in an injustice society, and this aspect shall not be disregarded. A distinction from the ethical discussion on encryption and DNS Privacy is that DNS operators are obliged to retain its logs for criminal investigation purposes, and therefore, encouraging the privacy is difficult to be seen as promoting illegal activities. 

\subsection{Recursive resolver centralisation}
E. Nygren from Akamai Technologies expressed concerns in the tendency of increased use of \textit{Public resolvers}.
He claimed that the tendency has `the risk of consolidating key parts of the Internet to rely on few services' and resulting in `significantly impaired Internet performance' for some use cases when ECS information is chosen not to sent \cite{akamai-dns-architecture}.

\subsubsection{DNS Privacy promoting the use of Public DNS servers}
It is however questioned whether the DNS Privacy technologies are encouraging users to chose the public recursive resolver in favour of the most popular Recursive resolver operators.
Due to difficulty in the manual configuration of TRR, it is likely for users to choose the default TRR, which are the \cite{dnsprivacy-test-servers} for Android 9's DoT implementation \cite{android-pie-dot} and CloudFlare in case of Firefox.
On the other hand, if more Internet service providers were supporting DoH or DoT resolvers and if these providers were gaining the users' trust, there is no necessary correlation for users deliberately changing to the public resolvers instead of utilising ISP provided resolvers upon DHCP standardisations \cite{peterson-doh-dhcp-00, peterson-dot-dhcp-00} are finalised and configuration are in practice.

\subsubsection{Allegations on performance decrease}
Not providing ECS information could lead to impaired performance of the Internet service because insufficient information on the geographical location of users misleads the choice of `topologically localised \cite{kintis2016understanding}' server for serving the client.

However, before blaming users choosing ECS-turned-off recursive resolvers or Recursive DNS operators not supporting the ECS, non-availability of clients for opt-outing of ECS from the users' perspective should be addressed.
According to Kintis et al., ``nearly five years after the ECS draft was proposed, there are still no client-centric tools that empower users to control how much of their IP address is revealed \cite{kintis2016understanding}''.
It is anticipated that once users gain control of how much of their IP to truncate, the decision on whether to sacrifice the availability for confidentiality shall rely on the users. It is not the content provider's obligation to prohibit Internet users from making their choices for the sake of the performance.

\subsection{IP as a human identifiable information}
People against using ECS often argue that ECS may reveal an IP address of the end-user behind the DNS query, and therefore has a chance of revealing the person's privacy.
However, it is an ongoing debate of whether the truncated IP address itself should be seen as a person identifiable factor.

Clent's IP address has a potential to be privacy harm, but main factors that contribute to the privacy infringement may be associated with the distinctness (uniqueness) of the address and whether the IP can be linked with other identifiers.

\subsection{QNAME minimisation}
Software BIND, the most commonly deployed software for DNS servers, supports QNAME minimisation by default from version 9.14.0 \cite{bind9qname}.
It is a significant achievement of DNS Privacy field, considering the market dominance of the BIND. Since the technology is enabled by default upon the updates, deploying the privacy technique is seen as ease for the recursive DNS server administrators.

When it comes to choosing which DNS recursive resolver server to use, DNS server operators and literature on DNS Privacy often recommend examining whether the server does not support ECS or not. Since the implementation of QNAME minimisation is officially in place, examining whether an arbitrary DNS Recursive resolver supports QNAME minimisation or not could also be a parameter to consider in choosing the TRR.

\subsection{Vulnerability of Tor}
The current study presented DNS-over-Tor as a possible solution for addressing the limitations of current DNS Privacy technology. However, utilising Tor implies that such practice inherits the vulnerabilities of the Tor as well. 

As an example of the vulnerabilities of the Tor network, Johnson et al. presented a study about Tor being susceptible to correlation attacks \cite{Johnson2013}.
In other words, if the entry-node and the exit-node of a Tor circuit were operated by the same party, of the metadata of the communication was shared by these parties, correlating the circuit session is feasible.
The study concluded that an adversary could deanonymise any given user who uses Tor regularly with over 50\% of probability within three months and over 80\% within six months.

This fact raises a concern to see whether applying Tor for DNS traffic would be an adequate choice or not, considering the technology's high cost of variable latency.
Nevertheless, DNS operations over Tor posses slightly different characteristics than correlating any regular web browsing because the portion of network traffics generated by DNS Query is much less compared to the portion created in loading websites.
For an arbitrary attacker to correlate DNS queries over Tor, it would require sophisticated efforts. Furthermore, the temporal cost of performing the attack is costly.
If an attacker managed to correlate the Tor traffic over three months, the value of information might not be valuable anymore.
However, to draw the worthiness of the attack, log retention period of subscriber information and DHCP allocation of the client's Internet Service Provider needs to be considered.
\newpage

\section{Conclusion}
The report presented an overview of privacy breaching scenarios of different end-user types.
While the majority of privacy literature focuses on the impact on individuals, the paper has attempted to represent organisational interests.

In addition to the investigation of different end-users' privacy interests, the report also presented a systematic survey of DNS Privacy enhancing techniques.

The report presented the limitations of the current DNS privacy technologies and explored the alternative approaches to overcome the limitations from the current techniques, which was to use Tor and to change a trusted recursive resolver frequently.

The end-users, whose interests are analysed in the report, could refer to the report to meet their interests.
The approach used in the analysis could be referred to in examining other emerging innovations in DNS Privacy. 

\subsection{Scientific contribution}
Although there exist other studies of a thorough investigation of DNS Privacy techniques, this report distinct by being transparent of the survey method and having provided approaches to categorise the DNS Privacy methods.

\subsection{Current milestone}
The report presented the status of client implementations of different DNS Privacy techniques and progress on the internet standardisations.
Although certain limitations of DNS Privacy methods exist as presented earlier, there is no doubt that DNS Privacy is cruising.

\subsection{Future work}
Due to the limitation of the time, the proof of concept proposal made in this study has not thoroughly examined.
The actual implementation of the proposal, especially DNS-over-Tor on Phase 2 and examining privacy contribution of using multiple trusted resolver can be done as future studies.
More specifically, future studies could explore the correlation of the privacy contributions and having different intervals for configuring Secure DNS servers (Trusted Recursive Resolvers).
\newpage


%----------------------------------------------------------------------------------------
%	References. IEEE style is used.
%
%----------------------------------------------------------------------------------------
\newpage

\hypersetup{urlcolor=black}
\bibliographystyle{IEEEtran}
\bibliography{references}
\newpage
%----------------------------------------------------------------------------------------
%	Appendix
%-----------------------------------------------------------------------------------------
\pagenumbering{Alph}
\setcounter{page}{1} % Reset page numbering for Appendix
\appendix

\section{Appendix 1}\label{scriptcode-appendix}
The section presents a set of Python script that can potentially be used in verifying the proof of concept presented in this study.
Sections follow with the code for processing a web site list, script to simulate the web traffic and the common base code for the above functions to work.
Analysing and capturing the generated web traffic is not in the scope of the section.

\subsection{Processing frequently visited web domain list} \label{processweblist}
The website list is fetched from Alexa top one million global charts and further classified depends on Top-Level-Domains (TLDs).
Below is code for a script to convert the Alexa list into a dictionary format.
\inputpython{../Selenium/process_web_list.py}{1}{30}

\subsection{Script for automating the web traffic simulation}
Selenium is the web automation test tool \cite{holmes2006automating}, which is typically used to test web applications. Selenium can be used to visit a list of websites for simulating DNS queries. In the script below, the selenium was incorporated with Firefox Gecko driver to control the web browser through a Python script.

Python 3.5 and higher, Pip3 is required. It is anticipated that Python packages such as selenium and json are also installed on the system. A Gecko driver needs to be reachable in the OS' PATH environment. This code assumes that the Firefox browser is installed on the local PC.

\inputpython{../Selenium/visitwebsites.py}{1}{70}
\subsection{Common based source}
Below is code for util.py which is a common base script.
\inputpython{../Selenium/util.py}{1}{110}
\newpage

\section{Appendix 2}\label{protocol-verification}
This section describes an experimental setup of verifying a DNS-over-HTTPS client.

\subsection{Experiment background}
Identifying incoming and outgoing traffics around a recursive resolver is the interest, as recursive DNS resolver knows about stub resolver's information and its query.
The location choice of a recursive resolver leads to different value addition of privacy enhancement as described in Section \ref{rr-location}. In each experiment, it is aimed to verify situations where  privacy enhancements are anticipated.

\subsection{Experiment setup}\label{simulation}
The remote recursive resolver (i.e. public DNS server) is chosen to be used as the focus of the experiment laid on observing on the wire traffic between the stub client and the resolver (i.e. \textbf{Phase 1}).

DoH clients that were available to test are Firefox web browsers later than version 62 \cite{FirefoxDoH}, curl later than 7.62 \cite{CurlDoH} and DNSCrypt-proxy.
Firefox's DoH client is chosen to be evaluated, and DNS traffics are generated in two ways:
\begin{enumerate}
    \item Automation of website visit using Selenium over Gecko driver.
    \\i.e. sequentially visiting a list of websites
    \item Simultaneous visit of the list of websites from the bookmark.
\end{enumerate}
Following configuration are made on Firefox:
\begin{enumerate}
    \item Set trr.mode as 2
    \\i.e. Use DNS over HTTPS first. If fails, use the conventional DNS
    \item Set trr.mode as 3
    \\i.e. Use DNS over HTTPS only
\end{enumerate}
Per each experiment, the following steps are done to minimise possible caching-effect:
\begin{lstlisting}[basicstyle=\ttfamily]
    # sudo systemd-resolve --flush-caches
    # sudo systemd-resolve --statistics
    $ wireshark -i enp4s0 -k & python3 visitwebsites.py
\end{lstlisting}
\subsubsection{Hypothesis}
Regardless of the chosen way to generate web traffic and regardless of the mode two or three of Firefox DoH operations, all DNS traffics will not be sent over the clear-text. Therefore, tcpdump and Wireshark packet captures of each experiment set will not reveal DNS query by observing UDP 53 port.

\subsection{Results}
When Selenium was used, configuration (1) with DoH first mode resulted in revealing all DNS queries, and DoH engine was not used.
When Selenium was used, configuration (2) with DoH only resulted in not being able to fetch any websites.
When bookmark-method was used, configuration (1) with DoH only resulted in some DNS traffics being revealed with conventional methods.
When bookmark-method was used, configuration (2) with DoH only resulted in not being able to fetch any websites.

\subsubsection{Firefox's DNS Lookup log}
While performing the experiments, the DNS related log files from Firefox were collected.
The log in debug level contained lines of `D/nsHostResolver TRR lookup url.tld' message followed by 'OK' or 'FAILED'.
Of the 638 lines, 407 entries lead to OK, and  231 entries indicated FAILED.
Approximately 63.79\% of the queries went through DNS-over-HTTPS.

\subsection{Evaluation}
This section provides an interpretation of the observed results.

\subsubsection{Possible interference caused by web automation}
Although Appendix \ref{scriptcode-appendix} discussed the potential usefulness of the web automation tool, using Selenium with Gecko Firefox webdriver resulted that DNS-over-HTTPS were not being used.
Since additional software is involved for a traffics simulation, there is higher a chance that an error in the automation tool may lead to misleading results, considering that bookmark-visit induced method did not show the same result.

\subsubsection{Bootstrap procedure}\label{bootstrap}
The reason that `DNS-over-HTTPS only mode' failed to load any website is assumed as the DoH engine failed to initiate the connection to the DoH Server URI of '\url{https://mozilla.cloudflare-dns.com/dns-query}'. It is interpreted as the experiments have not taken into the consideration of setting up the bootstrap procedures.

\subsubsection{Unexplained causes of DoH query failures}
There is no clear explanation on the reason of having 36.2\% of queries failed when DNS-over-HTTPS resolver was used.
The author of the DoH engine has not given a clear explanation either for causes leading to the failures \cite{daniel-doh}.
However, it is noteworthy to consider that simulation of web visits are made on simultaneous attempts; browsing twenty websites were initiated at the same time.
The simultaneous visit could have been a factor of leading to an overload situation.

\subsection{Conclusion of the DoH Experiments}
For the given experiment setups, using DoH over Firefox resulted in that some of DNS queries were leaked over conventional DNS resolver because DNS resolutions over its DoH engine failed.
\end{document}
