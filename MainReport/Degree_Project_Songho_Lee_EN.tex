%----------------------------------------------------------------------------------------
%
% LaTeX-template for degree projects at LNU, Department of Computer Science
% Last updated by Johan Hagelbäck, Mar 2017
% Linnaeus University
%
% License: Creative Commons BY
%
%----------------------------------------------------------------------------------------

%----------------------------------------------------------------------------------------
%	Settings and configuration
%----------------------------------------------------------------------------------------

\documentclass[a4paper,12pt]{article}

\usepackage[T1]{fontenc}
\usepackage{times}
\usepackage[english]{babel}
\usepackage[utf8]{inputenc}
\usepackage{dtklogos}
\usepackage{wallpaper}
\usepackage[absolute]{textpos}
\usepackage[top=2cm, bottom=2.5cm, left=3cm, right=3cm]{geometry}
\usepackage{appendix}
\usepackage[nottoc]{tocbibind}
\usepackage[colorlinks=true,
            linkcolor=black,
            urlcolor=blue,
            citecolor=black]{hyperref}

\setcounter{secnumdepth}{3}
\setcounter{tocdepth}{3}

\usepackage{sectsty}
\sectionfont{\fontsize{14}{15}\selectfont}
\subsectionfont{\fontsize{12}{15}\selectfont}
\subsubsectionfont{\fontsize{12}{15}\selectfont}
\usepackage{placeins}
\usepackage{csquotes} % Used to handle citations
\usepackage{pythonhighlight}

\renewcommand{\thetable}{\arabic{section}.\arabic{table}}  
\renewcommand{\thefigure}{\arabic{section}.\arabic{figure}} 

%----------------------------------------------------------------------------------------
%	
%----------------------------------------------------------------------------------------
\newsavebox{\mybox}
\newlength{\mydepth}
\newlength{\myheight}

\newenvironment{sidebar}%
{\begin{lrbox}{\mybox}\begin{minipage}{\textwidth}}%
{\end{minipage}\end{lrbox}%
 \settodepth{\mydepth}{\usebox{\mybox}}%
 \settoheight{\myheight}{\usebox{\mybox}}%
 \addtolength{\myheight}{\mydepth}%
 \noindent\makebox[0pt]{\hspace{-20pt}\rule[-\mydepth]{1pt}{\myheight}}%
 \usebox{\mybox}}

%----------------------------------------------------------------------------------------
%	Title section
%----------------------------------------------------------------------------------------
\newcommand\BackgroundPic{
    \put(-2,-3){
    \includegraphics[keepaspectratio,scale=0.3]{img/lnu_etch.png} % Background picture
    }
}
\newcommand\BackgroundPicLogo{
    \put(30,740){
    \includegraphics[keepaspectratio,scale=0.10]{img/logo.png} % Logo in upper left corner
    }
}

\title{	
\vspace{-8cm}
\begin{sidebar}
    \vspace{10cm}
    \normalfont \normalsize
    \Huge Bachelor Degree Project \\
    \vspace{-1.3cm}
\end{sidebar}
\vspace{3cm}
\begin{flushleft}
    \huge Effective utilisation of DNS Privacy\\ 
    \it \LARGE - Protection towards pervasive surveillance 
\end{flushleft}
\null
\vfill
\begin{textblock}{6}(10,13)
\begin{flushright}
\begin{minipage}{\textwidth}
\begin{flushleft} \large
\emph{Author:} Songho Lee\\ % Author
\emph{Supervisor:} Ola Flygt\\ % Supervisor
%\emph{Examiner:} Dr.~Mark \textsc{Brown}\\ % Examiner (course manager)
\emph{Semester:} VT 2019\\ % 
\emph{Subject:} Computer Science\\ % Subject area
\end{flushleft}
\end{minipage}
\end{flushright}
\end{textblock}
}

\date{} 

\begin{document}
\pagenumbering{gobble}
\newgeometry{left=5cm}
\AddToShipoutPicture*{\BackgroundPic}
\AddToShipoutPicture*{\BackgroundPicLogo}
\maketitle
\restoregeometry
\clearpage
%----------------------------------------------------------------------------------------
%	Abstract
%----------------------------------------------------------------------------------------
\selectlanguage{english}
\begin{abstract}
\noindent Current usage of the DNS system is the most significant loophole of Internet users' privacy, as all queries and answers for resolving web address are not protected in most cases. % Thesis statement, background description
Lack of confidentiality in the DNS system enables everyone who has control of the path between a user and DNS resolver can collect someone's usage pattern and fingerprint him and filter his access to specific websites. % Motivation
Despite a single solution for addressing privacy risks in all stages of the DNS query process does not exist, the report acquaints several Internet standardisations for DNS privacy that are complementary to the existing DNS system and verifies that implementation of these brings significant enhancement of users' privacy.
%The report explores existing methods to enhance DNS Privacy and sets up a series of experiments to verify implementations of such methods for privacy enhancement. % Description of problem explored
\\\\
\textbf{Keywords: DNS, DNS-over-HTTPS, DNS-over-TLS, DNS Privacy}
\end{abstract}

%----------------------------------------------------------------------------------------
%	Preface
%----------------------------------------------------------------------------------------
%\newpage
%\textbf {\large{Preface}}\\

%\noindent You can have a preface in the report if you want, but it is not necessary. In this you can write more personal reflections on your degree project. In the preface you can also take the opportunity to thank the people who have been particularly helpful during the report writing, for example if you had any contact with a company that helped with the project, people that guided or helped you during the project, or your family and friends that supported you during the project. The preface shall not be longer than half a page.

%----------------------------------------------------------------------------------------
\newpage
\pagenumbering{gobble}
\tableofcontents % Table of contents
\newpage
\pagenumbering{arabic}

%----------------------------------------------------------------------------------------
%
%	Here follows the actual text contents of the report.
%
%----------------------------------------------------------------------------------------

\section{Introduction}
This chapter describes what Doname Name System (DNS) is, and how the legacy design of DNS has become a privacy threat. Before discussing the privacy risks of DNS, the background section introduces relevant structure and mechanisms. Knowledgable readers in DNS and Client subnet function may go to section \ref{problemformulation}.

\subsection{Background}
Digital transformation has brought things used to be done in real life decades ago to the online. At work, people have a video conference call instead of a business trip if not necessary. For shopping goods, people fill in credit card numbers for cross-border payments rather than visiting a bank branch to issue paychecks. In other words, the stage where people work now has been shifted to cyberspace in recent decades, and the Internet has become an essential part of everyday life.

Despite the ubiquitousness of the Internet, users' activities online are collected under pervasive monitoring by different actors.
Pervasive monitoring means ``widespread attack on privacy \cite{rfc7258}.'' Information collected in such action could lead to a breach of users’ privacy, by re-identifying users based on traffic \cite{herrmann2010analyzing}, or could become aids for launching an active form of attacks, such as masquerade and Denial of Service (DoS).
Unfortunately, the insecure architecture of Domain Name System allows the pervasive monitoring, and thus it should be mitigated. Before discussing the privacy problems of DNS, we introduce DNS and its components which are important to address.

\subsubsection{DNS}
Every activity on the web most likely begins with entering a human-friendly domain name in the web-browser. Once we enter a domain name for visiting a website, DNS resolves the address to an actual Internet Protocol Address of a web server which hosts the website. In case multiple websites are hosted on a single server, the entered fully qualified domain name(FQDN) is used to differentiate virtual hosts on a web server \cite{virtual24host}. Therefore, DNS is a critical component of the Internet.
%What about describing hjierarchical structure of Domain Name System here?
\subsubsection{DNS Servers}\label{dnsservers}
DNS servers consist of four types: Stub resolver, Recursive resolver, Authoritative server, and Forwarding DNS server. Resolvers refer to programmes that obtain information from name servers upon clients' requests \cite{rfc1034}.

Stub resolver is a resolver that serves as an entry-point of querying DNS from applications and directs search request to the nearest recursive resolver \cite{rfc1123}. As it cannot complete domain name resolution by itself, stub resolver is dependant on a recursive resolver \cite{rfc8499}.

The recursive resolver is a server which receives a DNS query from a stub resolver and gets the final answer to the query, by (1) answering from its local cache or (2) sending queries to other DNS servers \cite{rfc8499}. After a recursive resolver has sent a query request to other authoritative name servers, it is expected for the resolver to store the answer as a \textbf{local cache}. It is the first server in DNS query flow that contacts other servers to get the answer for the client. 

Authoritative (name) server is a server that has ``authority over one or more DNS zones \cite{rfc8499}'' and ``can answer queries without needing to query on other servers as it knows the content of the queried DNS zone by local knowledge \cite{rfc2182}.''

DNS forwarding server is a server that forwards queries to recursive resolver or other forwarding servers. It does not perform a query process for the stub resolver.
\subsubsection{DNS Query process}
Due to the hierarchical structure of the Domain Name System with delegations of authorities \cite{rfc1591}, getting the exact IP address of a given domain name involves several DNS servers. Figure \ref{queryprocess} shows an example of querying ``saimei.ftp.acc.umu.se.''. 
\begin{figure}[ht!]
    \begin{center}
        \includegraphics*[width=\columnwidth]{img/dnsquery}
    \end{center}
    \caption{DNS Query sequence diagram}
    \label{queryprocess}
\end{figure}
In the diagram, steps 2 and 3 returns top-level-domain(TLD) from the root servers. The steps 4 and 5 obtain the Authoritative name server of Swedish TLD. The .SE TLD returns name server of Ume\aa\ University in steps 6 and 7. In the last, the name server of umu returns IPv4 address (A record) of the given address, so that recursive resolver can provide the answer to the stub resolver. These steps are performed under the assumption that none of the queries is cached. 
\subsubsection{EDNS(0) and Client Subnet}
The extension mechanisms for DNS (EDNS) is specified in RFC 6891. EDNS allows both DNS servers and client to send ``larger DNS packet than the original 512 octet limit \cite{rfc6891}'' so that it benefits of utilising larger size. It makes sending long IPv6 address and possible DNSSEC signatures. As of February 2019, major DNS resolver operators have started not to support non-EDNS compliant servers. 

EDNS(0) provides several options, and one of the options is Client Subnet(ECS) feature, as described in RFC 7831 \cite{rfc7871}. When ECS is used, recursive DNS servers provide a truncated client IP address in its DNS queries to the upstream authorities to permit ``topologically localised answers for Content Delivery Networks (CDN) \cite{kintis2016understanding}''.

\subsection{Related work}
The project accredits pioneer research of ``DNS Privacy Considerations (RFC 7626) \cite{rfc7626}'' which has provided theoretical foundations in analysis of DNS privacy. 
As of 2019, there are several studies presented to mitigate privacy issues based on the analysis. These studies are later presented in Chapter \ref{surveyresults}.

P. Werneck and J.H.C. van Heugten presented surveys of DNS privacy-enhancing methods. P. Werneck evaluated approaches to improve the privacy of DNS and stated the limitations of identified approaches \cite{werneck2014dns} in 2014. J. Heugten evaluated existing solutions to enhance DNS Privacy in 2018 in his study \cite{van2018privacy}. As standardisation of DNS-over-HTTPS is recently finalised \cite{rfc8484}, a study from van Heugten reflects more recent changes.

\subsection{Problem formulation}\label{problemformulation}
Currently, almost all DNS traffic is sent in clear text \cite{rfc7626} over the UDP protocol \cite{tcp2014analysis}, and it makes DNS queries vulnerable to being hijacked or used to filter users' traffic.
All participants of the DNS query process, as illustrated in Figure \ref{queryprocess}, transmit messages intensively, and these are in plaintext.
It is also noteworthy that all participating authoritative name servers receive the same questions, although it is not necessary for Authoritative servers in higher hierarchies in the process to know all the complete domain address in question.

S. Bortzmeyer has analysed that particular fields in DNS packet \cite{rfc1035} such as Query name (QNAME) and Source IP address reveal ``communication relationships \cite{rfc7626}''.
These series of observations indicate that there are risks of information leakage in following places: (1) tapping on the wire ``between the stub resolvers and the recursive resolvers'', and (2) information leaks in the servers.
The following research questions are formulated having regards to the privacy breaching circumstances.

\begin{table}[h!]
    \begin{tabular} {|p{1.2cm}|p{12.8cm}|} \hline
        \textbf{RQ1} & Which benefits would DNS Privacy bring to different parties? \\ \hline
        \textbf{RQ2} & Which are the use cases that the exisiting DNS Privacy enhancement methods could not address? \\ \hline
        \textbf{RQ3} & Which combination of technologies would be suitable to address the limitation of the current status?\\ \hline
    \end{tabular}
    \caption{Research questions}
\label{researchquestions}
\end{table}

\subsection{Motivation}
Most of the internet activities begin with DNS query, hence DNS is vital. Notwithstanding the importance of DNS, designers of the current DNS protocol have not taken consideration of ``confidentiality of protocol metadata'' \cite{wachs2014censorship}.
Therefore DNS queries reveal communication flows, and this property of DNS protocol is used in different contexts by different actors. Examples of usages are traffic monitoring for network management or limiting the influence of malicious websites by DNS Footprinting of malware \cite{stoner2010dns}, or detecting malware infections \cite{lemos2013got}.

Other exemplary usages of this property of DNS are nation-state surveillance \cite{NSA-SIGINT}, privacy-unfriendly activities of commercial sectors \cite{weaver2011redirecting}, and illegal actions by criminals. Surveillance affects individuals to possess stress and anxiety \cite{oulasvirta2012long}, and behavioural changes like self-censorship \cite{rfc6973}. RFC 6973 connotes that Privacy harms involve ``harms to financial standing, reputation, solitude, autonomy, and safety \cite{rfc6973}'' of individuals.

S. Farrell et al. state in RFC 7258 that allowing monitoring by benevolent actors and defending privacy against nefarious actors do not hold hand in hand, as the actions required to achieve both, regardless of the motivations, are indistinguishable \cite{rfc7258}.
Disadvantages incurred by lack of DNS privacy significantly overweight advantages, and therefore DNS privacy should be mitigated in any feasible practices.

\subsection{Objectives}
The following objectives are set to answer the research questions and transform the tasks into the smaller pieces, and reasonings on each objective follow below. Hereafter several words are shortened for the ease of denotation, such as an objective as O and a research question as RQ. Table \ref{objectives} summarises the objectives.
\begin{table}[h!]
    \begin{tabular} {|p{1.2cm}|p{12.8cm}|} \hline
        \textbf{O1} & Analyse different end-users' privacy infringements when not having DNS Privacy\\ \hline
        \textbf{O2} & Explore the state of arts in mitigative methods to enhance DNS Privacy \\ \hline
        \textbf{O3} & Identify areas which the selected methods could not address. \\ \hline
        \textbf{O4} & Analyse alternative approaches to mitigate the current limitations.\\ \hline
    \end{tabular}
    \caption{Objectives}
    \label{objectives}
\end{table}

RQ1 aims to explore benefits by having DNS privacy on a different group of people.
As it could be challenging to motivate benefits of having privacy when obscurity prevails, O1 is defined to analyse scenarios when the privacy is violated caused by having insecure or exposed DNS queries.

RQ2 is to find use cases of the existing DNS privacy mitigative methods. However, in order to discuss use cases, the state of these methods need to be examined. Therefore, O2 is set to explore the state of arts of the methods and O3 is set to identify areas for  improvements.

RQ3 attempts to provide useful interventions by finding a combination of technologies to overcome the limitation of the current DNS Privacy enhancement. For serving the purpose, O4 discusses possible approaches to using other technologies.

\subsection{Scope/Limitation}
The project has a focus on improving the privacy part, from the security perspectives. In other words, reflected to a Confidentiality, Integrity, Availability(CIA) triad, enhancing Integrity and availability perspectives are less prioritised in the current project. Issues and challenges of DNS security as a whole may be found in other studies, such as one conducted by Ning Hu et. al.\cite{ning2017dnssecurity}. 

\subsection{Target group}
The project aims to provide insight on DNS Privacy for Internet users and recursive resolver providers for improving users' privacy.
%Here you outline which target group that might be interested in your work. If you, for example, do a project about software architectures, a target group can be professional developers and architects that work with similar software systems as the system you investigated.

\subsection{Outline}
The majority of scientific reports follow in Introduction, Methods, Results, Analysis and Discussion(IMRaD) pattern. However, the project chose to use a design science method and therefore it has an inverted structure of results and analysis. An analysis is made first to describe the design constraints and results (or design proposal) are presented afterwards. 

\newpage
\section{Method}\label{Method}
This chapter describes the chosen scientific methods to answer the research questions (Table \ref{researchquestions}) and meet the objectives (Table \ref{objectives}).
The study used scientific methods of systematic literature review and design science. Controlled experiments were also partially used to verify several statements made in the study. Sections follow to motivate the choice of a scientific method for meeting objectives.
\subsection{Systematic literature review}
A systematic literature review was performed to accomplish O2, which is to study mitigative methods of DNS privacy. Although a recently published article provided an overview of DNS Privacy enhancing methods \cite{van2018privacy}, performing systematic literature review was necessary to eradicate possible biases and to minimise opportunities of missing suitable solutions.

A search criterium was set to list articles that cited RFC 7626 from a database Google Scholar to make the review process systematic. RFC 7626 is chosen, as its analysis had provided a clear insight of DNS privacy issues \cite{rfc7626}, and since around four years had passed after its publication, it was anticipated that fellow researchers have tried to solve or list risks identified in the article.

For inclusion criterium, references of the found articles were further examined, as there were chances of missing to address well-established solutions possibly because the methods had been introduced before publishment of RFC 7626.

Exclusive criteria were set to have the contents of the articles to be relevant to the defined problem. Therefore, any solving other security aspects of DNS, such as availability but not addressing the privacy problems were excluded.

\subsection{Design Science}
Design science ``creates and evaluates Information Technology artefacts to solve identified organisational problems \cite{von2004design}'', and it has strong relevance in addressing O3 and O4 which are to identify the area that studied DNS Privacy methods could not address and analyse alternative approaches to overcome the limitations.

To refine problem statements for the existing practice, a literature review is made to analyse privacy infringement scenarios on diverse user scenarios, as defined in O1.

\subsection{Controlled Experiment}
A controlled experiment is applied to verify whether the suggested design artefact addressed the limitations of current DNS Privacy enhancing methods found in O3 or not. Conducting a controlled experiment fulfils one of the guidelines of design science which requires ``thorough evaluation of the artefact \cite{von2004design}''.

\subsection{Reliability and Validity}
The study is seen to have reliability on the results of the literature review, as the same effect will be derived by performing a search as described in the previous section. As Appendix A includes the source code of the experiment scripts, a similar result is expected to be derived by other researchers as well. 

The project deployed its experiments in a virtualised environment to minimise unforeseen factors that would impact performance measurements.

\subsection{Ethical considerations}
Discussing ethical considerations has less significance in the chosen methods, as no real data of any physical persons is collected without the consent.
However, it can be questioned whether it is adequate to elaborate privacy breaching scenarios in details to use the information for formulating problems for the design science method.
Despite the concerns, as the project aims to improve the problematic scenarios of not having sufficient privacy enhancements, describing the problematic situations as-they-are is necessary.

\newpage
\section{Privacy profiles}
Privacy is seen as having an access control in confidentiality in a security domain.
In everyday language, Privacy means ``the right to control who knows certain things about you \cite{securityincomputing}''.
When it comes to information security, the special characteristic of information makes a ``propagation problem''.
It means that affected subjects lose control of the information about themselves after being disclosed.

Pfleeger introduces three aspects of Information security: sensitive data, affected parties, and controlled disclosure.
We will discuss privacy issues related to the DNS on different subjects.

\subsection{Affected subjects}
We classify subjects as private persons, and organisations. Organisations are further divided into a large organisation which operates own directory server with DNS, and the smaller organisations that do not operate DNS resolvers in its network.

\subsection{Sensitive information}
Defining what sensitive information is in subjective area.
Therefore, sensitiveness of the information cannot be measured in an absolute scale. However, several common area of sensitive information follows.

For natural persons, legally defined sensitive information are personal data revealing ``(1) ethnic origin, religious or philosophical beliefs, (2) trade-union membership, (3) health-related data, and (4) data concerning a person's sex life \cite{GDPR}''.

For organisations, of their information assets especially copyright (expression of the idea), trade secret, and privileged information may be seen as sensitive information \cite{securityincomputing}.

\subsection{Scenarios}
If a subject and its geographical location Frequencey of visit

\newpage
\section{Survey of DNS Privacy enhancing methods}\label{surveyresults}
\input{tex/survey_results.tex}

\newpage
\section{Analysis of DNS Privacy enhancing methods}
This section provides an analysis of the survey results.
The refined search results were categorised into the following fields: studies that suggested improving the security breach and investigations that demonstrated any application (or utilisation) of the security vulnerabilities.

\begin{figure}[h!]
    \begin{center}
    \includegraphics*[width=1\columnwidth]{img/dnsprivacy-classification}
    \end{center}
    \caption{Categorisation of DNS Privacy enhancing methods based on the approach}
    \label{dns-methods-classification}
\end{figure}

Studies that suggested improving the security breach (i.e. DNS Privacy methods) could be sorted to two approaches as Figure \ref{dns-methods-classification} demonstrated: ones which preserve the current hierarchical DNS structure and the other with the radical approach of proposing an architectural change to decentralised DNS.
Methods with preserving the hierarchical DNS had a tendency to prioritise interoperability with the existing domain infrastructure. Also, it was observed that in the case of applying channel encipherment solutions, methods tended to incapsulating DNS operations with minimal changes to the existing Internet protocols.

Classification of DNS Privacy methods could further be analysed based on the which privacy risk these address. S. Bortzmeyer identified risk area of DNS privacy in RFC 7626 as the followings \cite{rfc7626}: 
\begin{enumerate}
    \item Data in the DNS request
    \item On the wire
    \item In the servers
    \item Re-identification and other Interferences
\end{enumerate}
Following sections analyse privacy risk mitigations based on risk area and type of mitigation strategies as a whole. Concent, privacy and performance analysis of each mitigative method can be found in \cite{van2018privacy}.

\subsection{Encipherment of communication channels}
Communication channel encirpher methods, as presented in Table \ref{content}, attempt to alleviate risks on the wire. The methods also partially address the risks of re-identification depends on who the subject of the attacker is. 

\subsubsection{Two phases of DNS Query process}
For the conciseness of further analysis, the communication channel (transport channel) of DNS are abstracted into two phases or paths based on the DNS Query process.

\textbf{Phase 1} refers to step 1 and 10 of Figure \ref{queryprocess}. These two steps are performed in the DNS Query process when stub resolver queries a domain name to the recursive resolver and the recursive resolver replies to the stub resolver.
Channel-wise, it is denoted as a \textbf{stub-to-resolver link}.

\textbf{Phase 2} refers to the steps in the Figure \ref{queryprocess} where a recursive resolver finds the final answer to the queried address, by recursively reaching to concerning authoritative servers. In other words, the channel that the rest of the steps is performed is called \textbf{recursive-to-auth link}.

\subsubsection{Insufficient measurements on recursive-to-auth link}
The distinction of the two phases are noteworthy, as the current implementations that preserve DNS hierarchical structure do not secure all communication paths towards all involved parties of the DNS resolving.
As an example, the majority of the methods do not encrypt communications on Phase 2 (recursive-to-auth link) in contrast to Phase 1 (stub-to-resolver link).
The fact that an authoritative server having a one-to-many server-client relationship from the recursive resolvers is the major obstacle of applying encryption on Phase 2.

As a further explanation of one-to-many relationship being an obstacle, DoH \cite{rfc8484} and DoT \cite{hu2016specification} use Transport Layer Security (TLS) protocol \cite{rfc7858} for encryption. In the case an authoritative server process multiple TLS session, it is likely to end up exhausting its computational resources \cite{bhople2012server}, similar to a Distributed DoS (DDoS) attack situation. The internet draft ``Next step for DPRIVE: resolver-to-auth link \cite{I-D.bortzmeyer-dprive-step-2}'' discusses the aforementioned challenges.

Furthermore, authentication mechanisms are missing on Phase 2 \cite{I-D.bortzmeyer-dprive-step-2}, and the lack of authentication of the authoritative server may potentially enable a Man-in-the-middle attack (MITM).
Therefore, the location of the DNS resolver needs to be considered when to mention the limitations of each suggested methods.
% There are proposal of utilising TLS 1.3 and 

\subsubsection{Location of Recursive DNS resolvers}
From the end user's point of view, recursive DNS Resolvers can be on a local machine, one provided by the Internet Service Providers (ISP) and Public DNS servers \cite{van2018privacy}.
Selection of the location of recursive DNS resolvers leads to different impacts on the user's privacy, in terms of cache-sharing\cite{van2018privacy, wang2013analysis} and obfuscation and logging. Section \ref{dnsservers} described caching on recursive resolvers.

\begin{figure}[ht!]
    \begin{center}
    \includegraphics*[width=0.9\columnwidth]{img/local-recursive}
    \end{center}
    \caption{A simplified network map when using local resursive resolver server.}
    \label{localrecursive}
\end{figure}
When a user utilises a \textit{local recursive resolver} as illustrated in Figure \ref{localrecursive}, channel encipher methods do not add value to users' privacy considering that operations of phase two are often not encrypted.
Supposing that the user does not share local recursive resolver among the others, DNS queries which the user makes will not be fetched from a cache but, instead, from all involved Authoritative Name Servers (NS).
Sending queries in clear text on phase 2 leaves a possibility for all parties who are involved in the network packet transmission to monitor QNAME, query type and source IP of the traffic towards authoritative NS.
Referring to the assumption that local recursive resolver is unique for a person, there is no space for obfuscation since no one else is querying from the IP address of the recursive resolver.
However, utilising the local recursive resolver eliminates the risks of queries being logged (i.e. (3) risk in-the-servers) during Phase 1.


\begin{figure}[h!]
    \begin{center}
    \includegraphics*[width=0.9\columnwidth]{img/isp-recursive}
    \end{center}
    \caption{A simplified network map when using ISP-provided resursive resolver server.}
    \label{isprecursive}
\end{figure}
Using \textit{ISP provided Recursive resolver} is the most common scenario, as most ISP offer DNS resolver to their users by Dynamic Host Configuration Protocol (DHCP).
The resolver from ISP is shared with other subscribers of the network, and it increases more chance of having queries cached by another user who acquired the address previously.
Reusing cache reduces the need of Phase 2 in its response process \cite{wang2013analysis} and thus generates less `often-insecure' traffics towards authoritative NS. 
Authoritative name servers see the source IP address of ISP's resolver in Phase 2 of DNS resolving, but not IP of the individuals, in case E-DNS Client Subset(ECS) is not in place.
When ECS is used, the authoritative NS may see truncated IP of clients \cite{kintis2016understanding}, but the IP address does not present additional privacy harm as ISP's recursive resolver is often in the same subnet IP range, and authoritative NS already acquaints source IP of the recursive DNS resolver. 
Privacy risks incurred by logging may exist in ISP provided resolver, as ISPs may be obliged for log retentions due to legal requirements of the countries they operate in.
Channel encipherment on Phase 1 adds a value of users' privacy to a limited extent, tapping on the wire between the ISP's recursive resolver and a stub resolver is often feasible for ISP itself rather than third parties.

\begin{figure}[h!]
    \begin{center}
    \includegraphics*[width=0.9\columnwidth]{img/public-recursive}
    \end{center}
    \caption{A simplified network map when using public resursive resolver server.}
    \label{publicrecursive}
\end{figure}

A scenario of using a \textit{public recursive resolver} (as known as Public DNS resolver) is represented in Figure \ref{publicrecursive}.
A public DNS server has more possibilities of being shared by a broader public compared to the ISP provided resolvers, and it increases the chances of queries being already cached. Authoritative servers see requests from the IP address of the public resolver, instead of stub resolvers' when ECS is not applied.

This scenario benefits users the most when channel encipherment is applied because DNS query contents in Phase 1 is not visible for parties in the middle of the networking path. It brings significant obfuscation in tracking down the end-user by analysing the network traffics. However, Public DNS servers may log the DNS queries and information of the client. Therefore, privacy risk in-the-servers remains.

\subsubsection{Delegation of trust}
Applying channel encipherment methods for securing DNS queries such as using DNS-over-HTTPS or DNS-over-TLS implies that end-client gives trust in the chosen resolver.

\FloatBarrier
\subsection{Information redactions}
Applying the encipherment mechanism on the communications among DNS servers has limitations that these only shifted the trust towards the Recursive resolver.
In the meanwhile, design enhancements of packets' content reduce the risks of Data in the DNS request and data leaks in the servers.
Query Name (QNAME) minimisation \cite{bortzmeyer2016dns} and Oblivious DNS \cite{annee-dprive-oblivious-dns-00} are the techniques of this category.

\subsubsection{Information leak minimisations}
\textit{QNAME minimisation} reduces the query leaks on higher authoritative name server chains, by presenting only relevant part of a domain name for an authoritative server to answer, instead of querying with FQDN.
It makes only the relevant name server to acquire the full QNAME and query type. Therefore it reduces privacy risks on Phase 2.
Although the risk of data leaks still present in the last chain of the authoritative NS, this is less likely to happen unless administrators of the domain name have a malicious intent to exploit users' privacy.

\subsubsection{Decoupling user-identifiable information}
\begin{figure}[h!]
    \begin{center}
    \includegraphics*[width=0.9\columnwidth]{img/ODNSoverview}
    \end{center}
    \caption{An overview of Oblivious DNS \cite{ODNSwebsite}}
    \label{odnsoverview}
\end{figure}
\textit{Oblivious DNS (ODNS)} aims to decouple any association of a client IP address and DNS query content and no single party should be able to see both \cite{annee-dprive-oblivious-dns-00}.
For this to work, it requires a special stub resolver and an ODNS authoritative resolver where the client creates a unique session key on each session to encrypt its DNS query and append 'odns' TLD.
Figure \ref{odnsoverview} presents an overview of the ODNS and the illustration is fetched from its project page \cite{ODNSwebsite}.

However, security concerns arise in terms of availability, as ODNS operations require ODNS-stubs and imposing such design creates a new `central point of failure \cite{minutes-102-dprive}'.
Also, questions concerning assuring confidentiality of the keys used in ODNS and issues with fallback have not been answered \cite{minutes-102-dprive}. 

\subsection{Architectural shift}
In the previous section, securing recursive-to-auth link is seen challenging due to the hierarchical structure of the DNS.
Studies that aim towards decentralised, often imposing peer-to-peer (P2P) implementation are free from the limitations caused by the hierarchical structure.
Examples of such solution are Namecoin \cite{loibl2014namecoin} and GNU Name System (GNS) \cite{grothoff2017nsa, wachs2014censorship}.

\subsubsection{Blockchain}
\textit{Namecoin} \cite{loibl2014namecoin} is described as ``timeline-based system that relies on a P2P network to manage updates and store the timeline \cite{grothoff2017nsa}'', and it settles down any commits related on `key-value mapping' by transactions that are published in an append-only hash chain (a.k.a. blockchain) \cite{kalodner2015empirical}.
However, preserving user privacy in Namecoin requires ``replication of the full blockchain at the user's end system \cite{grothoff2017nsa}'' and performing such received a criticism that it ``may be impractical for some devices \cite{grothoff2017nsa}.''
The criticism can be interpreted as applying blockchain did not add significant value, in terms of preserving the privacy of users, as privacy breaches in DNS resolving process could also be addressed if DNS records could be replicated on a client's system even in the existing hierarchical system. 

\subsubsection{Distributed Hash Table}
\textit{GNU Name system (GNS)} is `privacy-preserving' domain name lookup system which takes a radical deviation from the conventional domain resolvers, by utilising P2P network and Distributed Hash Table (DHT).
DHT provides ``support for an operation: given a key, it maps the key onto a node \cite{stoica2001chord}'', such property is used in GNS domain lookup (resolving) process. 
In other words, GNS utilise ``distributed storage of DNS records in P2P overlay networks \cite{wachs2014censorship}''. In addition to the DHT, GNS is built upon a petname system \cite{stiegler2005introduction} and utilise Simple Distributed Security Infrastructure (SDSI).

Authors of the GNS claim that domain resolving process of the GNS is private because ``queries and responses are encrypted \cite{grothoff2017nsa, wachs2014censorship}.''
However, its confidentiality is vulnerable for `confirmation attack' if ``the adversary knows both the public key of the zone and the specific label \cite{wachs2014censorship}.'' Note that the label refers to an entry in a petname system. 

\subsubsection{Prevalence of hierarchical DNS}
Although the primary focus of the study is made on Domain Name resolving part, DNS resolving is only a portion of the entire DNS, as there are diverse actors (such as administrators, domain owners) involved in the system.
Figure \ref{dnsactors}, which is presented by SUNET \cite{SUNET-DNS}, illustrates the involved machines and servers of the system as a whole.

\begin{figure}[h!]
    \begin{center}
    \includegraphics*[width=1\columnwidth]{img/DNS-maskinvara}
    \end{center}
    \caption{Parties involved in DNS as a whole \cite{SUNET-DNS}}
    \label{dnsactors}
\end{figure}

In the figure, sections are vertically divided.
The figure in the first column represents organisations or individuals that desire to acquire a new domain name.
The third column of the figure demonstrates administrators of delegation on each autonomous domain hierarchy.
Legacy of the system are often neglected in attempts of securing the domain name resolution process, which is presented in green colour in the figure. 

The naming hierarchy of the DNS ``ties into systems such as the Public Key Infrastructure (PKI) \cite{akamai-dns-architecture}'', and architecture of decentralised DNS, such as NameCoin \cite{loibl2014namecoin}, may have not considered the PKI structure in its design.

\subsection{Constraints}
Up until the previous section, privacy-enhancing studies were analysed in terms of contribution to privacy enhancements on end-users.
The following section highlights factors other than the privacy (confidentiality) perspective.

\subsubsection{CIA-triad}
In information security discussions, threat mitigations of a system are analysed in three perspectives: confidentiality, availability and integrity. These properties as a group are denoted as CIA triad or the security triad.
Achieving every aspect of CIA-triad often are not feasible, as enhancing one dimension may interfere with the other dimensions \cite{securityincomputing}.

\subsubsection{The integrity of the DNS records}
Ensuring the integrity of the DNS record is significant, and it is evident from `phishing attacks \cite{ariyapperuma2007security, ollmann2004phishing}'.
Per contra, radical protocol design proposals may harm the integrity of the DNS records, as its proof-censorship or legal-attack-proof designs may enable everyone can claim the legitimacy of ownership of the specific domain.

Solutions using DHT in its design may put the system vulnerable to possible Sybil attacks \cite{6503215, SitE2002Scfp}.
Sybil attacks are a type of attack scenario where an attacker `subverts the reputation system by creating a considerable amount of pseudonymous identities \cite{TRIFA20141135}' `in the absence of identification authority \cite{douceur2002sybil}.'

Interventing the malvaceous records are more challenging than the hierarchical structure of the current DNS system in the decentralised design, and this could lead to concerns of enabling phishing attacks.
Enhancing confidentiality aspects of DNS security is important but it should not compromise the integrity aspect.

\subsubsection{The availability of the DNS Service}
There are several criteria to define availability. The criteria for defining availability introduced by Pfleeger are (1) having a timely response to a given request, (2) ensuring fair allocation of resources and (3) being fault-tolerant \cite{securityincomputing}.
This report defines availability as a chosen recursive resolver to resolve the queried address and provide the correct IP address promptly.

High availability of DNS is important since web activities cannot be initiated without name resolution (as mentioned in Section \ref{dns-introduction}). High performance (performance refers to having a low latency) of DNS is also an important factor to consider because DNS resolutions are a significant cause of having log Web waits \cite{cohen2003proactive, jung2002dns}, and it has a notable impact on `User-perceived latency'.
\newpage

\section{Design Constraints}
In the previous chapter, privacy-enhancing studies were analysed in terms of contribution to privacy enhancements on end-users.
In this section, factors other than the privacy perspective are highlighted.

\subsection{CIA-triad}
Achieving every aspect of CIA-triad often are not feasible, as enhancing one dimension may interfere with the other dimensions.

\subsubsection{The integrity of the DNS records}
Ensuring integrity of the DNS record is significant, and it is evident from `phishing attacks \cite{ariyapperuma2007security, ollmann2004phishing}'.
Per contra, radical protocol design proposals may harm the integrity of the DNS records, as its proof-censorship or legal attack proof designs may enable everyone can claim the legitimacy of ownership of the specific domain.
Furthermore, solutions using DHT in its design may put the system vulnerable to possible sybil attacks \cite{6503215, SitE2002Scfp}.
Interventing the malvaceous records are more challenging than the hierarchical structure of the current DNS system in the decentralised design, and this could lead to concerns of enabling phishing attacks.
Enhancing confidentiality aspects of DNS security is important but it should not compromise the integrity aspect.

\subsubsection{The availability of the DNS Service}
There are several criteria to define availability. The criteria for defining availability introduced by Pfleeger are (1) having a timely response to a given request, (2) ensuring fair allocation of resources and (3) being fault-tolerant \cite{securityincomputing}.
This report defines availability as a chosen recursive resolver to resolve the queried address and provide the correct IP address promptly.

High availability of DNS is important since web activities cannot be initiated without name resolution (as mentioned in Section \ref{dns-introduction}). High performance (performance refers to having a low latency) of DNS is also an important factor to consider because DNS resolutions are a significant cause of having log Web waits \cite{cohen2003proactive, jung2002dns}, and it has a notable impact on `User-perceived latency'.


\newpage

\section{Proposal}
The previous chapter provided a categorisation of the DNS Privacy technologies.
Afterwards, possible shortcomings of each type of technologies were mentioned.

For the designs using the channel encipherment, DNS Queries on Phase 2 were not sufficiently addressed due to the widespread authoritative name servers and resource exhaustion problem caused by many-to-one communication relationship.
Despite many operational authoritative name servers were deployed in the load-balanced and highly available manner and term `many-to-many' shall be a more accurate description of the current practices, the number of recursive resolvers prevails load balancing of each name domain.

Methods which reduced the information leak by redacting information (i.e. not showing unnecessary or irrelevant information for an authoritative server to answer) and methods with decoupling sensitive data did not have obvious limitation.

Proposals with Architectural transition raised concerns of harming other aspects of security triad such as integrity of the DNS records.

In the following sections, we reflect approaches to mitigate the addressed limitations. Limitations caused by architectural design is not focused. In other words, the scope lies in the approaches that preserve the current hierarchical DNS.

\subsection{Trusted recursive resolver}

\subsection{Non-trusted recursive resolver}
When a trustable recursive resolver does not exist, offen end-users' own computer becomes the recursive resolver. However, the earlier chapter analysed that utilising recursive resolver on a local machine barely gives any value, because many of recursive-to-auth links are unencrypted and subject to the traffic monitoring.

\subsubsection{Traffic anonymisation}
To circumvent the situation, traffic anonymisation technologies on recursive-to auth link can be applied, and examples of such technology are  FreeNet \cite{clarke2001freenet} or GNUNet\cite{grothoff2017gnunet}, and Tor.
However, FreeNet \cite{clarke2001freenet} or GNUNet \cite{grothoff2017gnunet} result in having high delays \cite{anonymousoverdns}.

Solution: Proactive caching \cite{cohen2003proactive} over Tor for most significantly visited websites. For caching, the same tor circut can be reused but for processing individual queries, connection shall not be resued.

Variation of Round Trip Time and its impact on the end-user's perseption shall be discussed. Intercoperation problem with CDN follows.
Provide Privacy analysis on the Tor. Tor may still be subject to the confirmation attacks, similar to DHT technologies. 

Tor cannot forward UDP traffic on the exit-node. (Citation needed). Therefore, it may be difficult to argure the planned idea.
Instead, try to combine DNS-over-TLS or DNS-over-HTTPS with Tor over its proxy socket \cite{tor-socks}, and try to use a DNS resolving client such as \cite{technitium-configuration}. 
\newpage
	
\section{Discussion}
The project intensively examined securing Domain Name queries as a method of enhancing end-users' privacy towards pervasive monitoring and presented sections for answering the defined objectives. In this chapter, arguments on the current direction of DNS Privacy methods developments are introduced. Arguments of whether it is ethical to empower DNS Privacy also follows.

\subsection{Privacy leaking components apart from the DNS}
A question may arise why it mainly focuses on securing DNS, although there exist other factors which disclose users' privacy.
Mechanisms of a web filter are examined to answer the question, as it is a commonly found practical example of the large scale monitoring \cite{murdoch2008tools}.

Web filter, also known as content-control software, is software that restricts access to a content that is delivered on the Web.
Wazen et. al categorise mechanisms of legacy web-filtering into five techniques: (a) Port-based, (b) DNS, (c) IP Address, (d) Certificate, (e) Payload-based (f) HTTP proxy filtering techniques \cite{shbair2015efficiently}.
Except for the technique based on DNS filtering, the rest methods are regarded as well-mitigated due to recent developments of the web environment. 

Among the various types of filtering techniques mentioned above, methods (a) and (c) are considered less efficient due to changes in the Internet ecosystem in recent decennial;
Internet firms such as Google, Facebook and Amazon show strong presence \cite{haucap2014google}, and the phenomenon may have reduced the diversity of traffic endpoint's IP addresses.
Moreover, it has become more common to have web services deployed in cloud environments \cite{clouds2018stat}, and IaaS providers extensively use `Virtual Host \cite{virtual24host}', which means various Web servers correspond to the same IP address.
It also eliminates the need for utilising different ports to co-host services. Thus, port usages are normalised.

Also, another notable change of the Internet is that adoption of HTTPS on the web has increased significantly \cite{felt2017measuring}.
The change has increased costs of performing technique (e) and brought challenges in payload-based traffic classification \cite{xue2013traffic}.
Also, it has made (f) less applicable, as a proxy does not directly process encrypted traffics \cite{shbair2015efficiently}.
Furthermore, the combination of wide deployment of HTTPS and Virtual Hosting has made technique (e) inefficient, because ``many companies share the same certificate across different services and domain names \cite{shbair2015efficiently}''.

However, the trend change of Internet has not brought additional challenges to Domain Name System (DNS) filtering. Therefore, the project studies to remedy the weakest point towards users' privacy, which in this case, is DNS.

\subsection{DNS Privacy - possible aids for the criminals}
At the beginning of the report, the thesis motivated that securing DNS in its architecture is necessary, as the vulnerability cannot only be exploited by the right hand but could also be misused by a malevolent party.
It is feasible that the perpetrators may use DNS Privacy enhancement methods to hide their activities, and DNS Privacy is in good practice that makes it difficult for investigators to eavesdrop the queries.

In the U.S. or common law jurisdictions, hindering lawful enforcement may be seen as ``Obstruction of justice''.
In this perspective, hiding DNS query by means of encryptions could be questioned whether such behaviour is \textit{concealing} or \textit{covering up} with the intent to \textit{impede} or \textit{obstruct} the investigation or proper administration as 18 U.S. Code \S 1591 states \cite{Obstructionofjustice}.

M Bay has performed a thought experiment of the relation of civil disobedience and unbreakable encryption \cite{bay2017ethics} based on John Rawls' theory of justice.
Not all DNS Privacy enhancing methods have direct linkage with the `unbreakable encryption' since encryptions are breakable with `the given sufficient time and computational resources \cite{ellison2000ten,chau2006application}'.
However, it is noteworthy to introduce the aspect as exercising DNS Privacy has similarity in a sense that DNS Privacy methods aim to achieve unbreakable-encryption-like situation.

M Bay had made three assumptions before applying Rawlsian principles to encryption as follows \cite{bay2017ethics}:
\begin{enumerate}
  \item Unbreakable/impenetrable encryption is indeed impenetrable
  \item The encryption in question is available to citizens and is not exclusive to certain institutions within society
  \item The society examined here can be described as well-ordered \cite{moon_2014, RawlsJohn1973Atoj} in Rawls’ terminology
\end{enumerate}
The thought experiment led to the following points: ``In a well-ordered society, an obstruction of law enforcement is at the same time an obstruction of principles of justice agreed upon by the society’s citizens.'' and  ``if society is just, citizens must comply with its institutions in order for it to remain just.''
However, in the case of ``societies are only partially just or in which, say, an unjust war is waged by an otherwise just society. Then, the basic liberties of the individual take precedence, in the push for the restoration of justice \cite{RawlsJohn1973Atoj}.''
However, according to Bay, civil disobedience and conscientious refusal lead to a violation of Rawls’ principles as he hypothesised with point three: the society is well-ordered.

With regards to applying utilitarian arguments for encryption, M Bay concluded as ``utilitarian reasoning does not provide us with a solution with regard to the conflict between encryption, privacy and enforcement of justice, since it simply becomes a version of the age-old conflict between security and freedom at a higher level of abstraction \cite{bay2017ethics}.''

Judging whether empowering DNS privacy is justice or not is a debatable question in many aspects. However, the encryption or having privacy facilities the freedom of speech in an injustice society, and this aspect shall not be disregarded. A distinction from the ethical discussion on encryption and DNS Privacy is that DNS operators are obliged to retain its logs for criminal investigation purposes, and therefore, encouraging the privacy is difficult to be seen as promoting illegal activities. 

\subsection{Recursive resolver centralisation}
E. Nygren from Akamai Technologies expressed concerns in the tendency of increased use of \textit{Public resolvers}.
He claimed that the tendency has `the risk of consolidating key parts of the Internet to rely on few services' and resulting in `significantly impaired Internet performance' for some use cases when ECS information is chosen not to sent \cite{akamai-dns-architecture}.

\subsubsection{DNS Privacy promoting the use of Public DNS servers}
It is however questioned whether the DNS Privacy technologies are encouraging users to chose the public recursive resolver in favour of the most popular Recursive resolver operators.
Due to difficulty in the manual configuration of TRR, it is likely for users to choose the default TRR, which are the \cite{dnsprivacy-test-servers} for Android 9's DoT implementation \cite{android-pie-dot} and CloudFlare in case of Firefox.
On the other hand, if more Internet service providers were supporting DoH or DoT resolvers and if these providers were gaining the users' trust, there is no necessary correlation for users deliberately changing to the public resolvers instead of utilising ISP provided resolvers upon DHCP standardisations \cite{peterson-doh-dhcp-00, peterson-dot-dhcp-00} are finalised and configuration are in practice.

\subsubsection{Allegations on performance decrease}
Not providing ECS information could lead to impaired performance of the Internet service because insufficient information on the geographical location of users misleads the choice of `topologically localised \cite{kintis2016understanding}' server for serving the client.

However, before blaming users choosing ECS-turned-off recursive resolvers or Recursive DNS operators not supporting the ECS, non-availability of clients for opt-outing of ECS from the users' perspective should be addressed.
According to Kintis et al., ``nearly five years after the ECS draft was proposed, there are still no client-centric tools that empower users to control how much of their IP address is revealed \cite{kintis2016understanding}''.
It is anticipated that once users gain control of how much of their IP to truncate, the decision on whether to sacrifice the availability for confidentiality shall rely on the users. It is not the content provider's obligation to prohibit Internet users from making their choices for the sake of the performance.

\subsection{IP as a human identifiable information}
People against using ECS often argue that ECS may reveal an IP address of the end-user behind the DNS query, and therefore has a chance of revealing the person's privacy.
However, it is an ongoing debate of whether the truncated IP address itself should be seen as a person identifiable factor.

Clent's IP address has a potential to be privacy harm, but main factors that contribute to the privacy infringement may be associated with the distinctness (uniqueness) of the address and whether the IP can be linked with other identifiers.

\subsection{QNAME minimisation}
Software BIND, the most commonly deployed software for DNS servers, supports QNAME minimisation by default from version 9.14.0 \cite{bind9qname}.
It is a significant achievement of DNS Privacy field, considering the market dominance of the BIND. Since the technology is enabled by default upon the updates, deploying the privacy technique is seen as ease for the recursive DNS server administrators.

When it comes to choosing which DNS recursive resolver server to use, DNS server operators and literature on DNS Privacy often recommend examining whether the server does not support ECS or not. Since the implementation of QNAME minimisation is officially in place, examining whether an arbitrary DNS Recursive resolver supports QNAME minimisation or not could also be a parameter to consider in choosing the TRR.

\subsection{Vulnerability of Tor}
The current study presented DNS-over-Tor as a possible solution for addressing the limitations of current DNS Privacy technology. However, utilising Tor implies that such practice inherits the vulnerabilities of the Tor as well. 

As an example of the vulnerabilities of the Tor network, Johnson et al. presented a study about Tor being susceptible to correlation attacks \cite{Johnson2013}.
In other words, if the entry-node and the exit-node of a Tor circuit were operated by the same party, of the metadata of the communication was shared by these parties, correlating the circuit session is feasible.
The study concluded that an adversary could deanonymise any given user who uses Tor regularly with over 50\% of probability within three months and over 80\% within six months.

This fact raises a concern to see whether applying Tor for DNS traffic would be an adequate choice or not, considering the technology's high cost of variable latency.
Nevertheless, DNS operations over Tor posses slightly different characteristics than correlating any regular web browsing because the portion of network traffics generated by DNS Query is much less compared to the portion created in loading websites.
For an arbitrary attacker to correlate DNS queries over Tor, it would require sophisticated efforts. Furthermore, the temporal cost of performing the attack is costly.
If an attacker managed to correlate the Tor traffic over three months, the value of information might not be valuable anymore.
However, to draw the worthiness of the attack, log retention period of subscriber information and DHCP allocation of the client's Internet Service Provider needs to be considered.
\newpage

\section{Conclusion}
In this chapter you end your report with a conclusion of your findings. What have you shown in your project? Are your results relevant for science, industry or society? How general are your results (i.e. can they be applied to other areas/problems as well)? Also discuss if anything in your project could have been done differently to possibly get better results. 

This chapter is also written in present tense.

\subsection{Future work}
Due to the limitation of the time, the proof of concept proposal made in this study has not throughouly examined. Actual implementation of the proposal can be done as a future study. Appendix presents a draft version of the test code to automate the web traffic generation.
\newpage


%----------------------------------------------------------------------------------------
%	References. IEEE style is used.
%
%----------------------------------------------------------------------------------------
\newpage

\hypersetup{urlcolor=black}
\bibliographystyle{IEEEtran}
\bibliography{references}
\newpage
%----------------------------------------------------------------------------------------
%	Appendix
%-----------------------------------------------------------------------------------------
\pagenumbering{Alph}
\setcounter{page}{1} % Reset page numbering for Appendix
\appendix

\section{Appendix 1}
The section presents a set of Python script that can potentially used in verifying the proof of concept presented in this sutdy.
Secetions follow with code for processing a web site list, script to simulate the web traffic and the common base code for above functions to work.
Analysing and capturing the generated web traffic is not in the scope of the section.

\subsection{Processing frequently visited web domain list} \label{processweblist}
The website list is fetched from Alexa top one million global chart and further classified depends on Top-Level-Domains (TLDs).
Below is code for a script to convert the Alexa list into a dictionary format.
\inputpython{../Selenium/process_web_list.py}{1}{30}

\subsection{Script for automating the web traffic simulation}
Selenium is the web automation test tool\cite{holmes2006automating}, which is typically used to test web applications. Selenium can be used to visit a list of websites for simulating DNS queries. In the script below, the selenium was incorporated with Firefox Gecko driver to control the web browser through a Python script.

Python 3.5 and higher, Pip3 is required. It is anticipated that Python packages such as selenium and json are also installed on the system. A Gecko driver needs to be reachable in OS' PATH environment. This code assumes that Firefox browser is installd on the local PC.

\inputpython{../Selenium/visitwebsites.py}{1}{70}
\subsection{Common based source}
Below is code for util.py which is a common base script.
\inputpython{../Selenium/util.py}{1}{110}
\end{document}
